% Chapter Template

\chapter{STIX} 
\label{Chapter3}
\lheader{Chapter 3. \emph{STIX}} % Change X to a consecutive number; this is for the header on each page - perhaps a shortened title

%----------------------------------------------------------------------------------------
%	SECTION 1
%----------------------------------------------------------------------------------------

\section{Background}

Las aproximaciones tradicionales para la seguridad, se focalizan en entender y 
registrar las vulnerabildes, debilidades y configuraciones necesarias pero 
insuficientes. Las defensas efectivas contra las amenazas actuales y futuras 
también requiere la adición sobre el comportamiento, capacidades e intenciones 
del atacante. Entendiendo al adversario y a nosotros podemos entender lo 
suficiente respecto a la naturaleza de las amenazas a las que se enfrenta la 
organización para tener decisiones para una defensa efectiva. El comportamiento 
de los adversarios no esta solamente focalizado de forma extendida en 
actividades destructivas, sino que también se focaliza en  objetivos de mas bajo 
nivel que apuntan a conseguir objetivos tacticos y establecer puntos de entrada 
a las organizaciones. 
//ACA SE PUEDE UNIR CON KILL CHAIN QUE ESTA EN TAXII
La cyber inteligencia busca entendenr y caracterizas las acciones que un 
atacante puede o podria realizar, como estas pueden ser detectadas y 
reconocidas, como pueden ser mitigadas y cuales son los actores relevantes, etc.
Un entendimiento de la amenaza que implica el adversario permite la realización 
de decisiones mas efectivas, la priorización de recursos y llegar a tener una 
oportunidad de tomar la ventaja ante el adversario. El efecto de las defensas en 
base a inteligencia es una mejor postura respecto a las respuestas dado que los 
atacantes ajustan sus operaciones basandose en el exito o falla de sus intentos. 
En un modelo de kill chain los intentos que realiza un adversario pueden ser 
reconocidos y logrando asi que los defensores tengan la posiblidad de ajustar 
sus tacticas para una mejor respuesta. Esto hace que al adversario le sea mas 
dificil alcanzar sus objetivos.
En esto se presenta un desafio dado que ninguna organizacion por si sola tiene 
acceso a un conjunto de información relevante para tener una identificación 
precisa de una amenaza. La forma de sobrepasar esta limitación es por medio del 
intercambio de información relevante sobre las amenazas con socios y comunidades 
de confianza. Por medio del intercambio de información, cada ente puede alcanzar 
un nivel mas alto de entendimiento del panorama de las amenazas, no solamente de 
forma abstracta sino que también que cosas especificas indican la presencia de 
una atacante.
Dada la forma en la complejidad con la que evoluciona el panorama de amenazas, 
la velocidad a la cual ocurren los eventos, y la basta cantidad de datos que se 
deberian intercambiar, es necesario establecer una forma automatica para ayudar 
a los humanos o para ejecutar acciones defensivas para que esta aproximación sea 
efectiva.
La automatizacion requiere de información de calidad y las mayoria de las 
capacidades defensivas son construidas con arquitecturas heterogeneas. La 
combinación de estos factores requiere estandarización, representaciones 
estructuradas de información y un aprovechamiento de la información sin saber de 
antemano quien va a proveer que información.
Un desafio que tienen las organizaciones a la hora de realizar intercambio de 
información es el de tener la habilidad de estructurar la información sin perder 
el juicio y control que puede tener un humano. La información intercambiada 
tiene que ser leible por un humano y parseable por una maquina. Este 
requerimiento es utilizado por porgramas de intercambio de información en los 
cuales las organizaciones no solo consumen los datos sino que también los 
evaluan como parte del proceso de inteligencia. Este proceso es llevado a cabo 
por analistas que se focalizan en tipos de analisis que son inapropiados para 
ser automatizados o focalizados en tomas de decisiones por parte de humanos en 
donde el analista lee la información para obtener información de la situación. 
Además, el analista esta constantemente evaluando la fidelidad de las fuentes y 
los métodos utilizados para producir la información.
Dados estos factores, es necesaria la existencia de representaciones 
estructuradas de la información y que esta información sea expresiva, flexible, 
extensible, automatizable y que se pueda leer. En este punto es donde surge STIX 
como una solución al problema presentado.

\section{Aproximaciones de la actualidad}
La información que se intercambia y utiliza hoy en día es atómica, 
inconsistente y muy limitada en sofisticación y expresividad. En donde se 
utilizan estructuras estandarizadas, estas son tipicamente focalizadas en una 
porción del problema. Además no se integran de forma adecuada entre si o carecen 
de flexibildiad. Las actividades de intercambio de indicadores son entre 
humanos, siendo los indicadores desestructurados o semi-estructurados 
intercambiados por portales web o encriptados via email. Más recientemente se 
ha visto el surgimiento de transferencias entre maquinas de conjutnso de 
indicadores simples de modelos de ataques bien conocidos.
STIX busca extender los indicadores para permitir el manejo e intercambio de 
indicadores de forma más expresiva así como de un espectro más amplio de 
información.
Actualmente, el intercambio y manejo de información de forma automatica de 
información es visto tipicamente en lineas de productos, servicios ofrecidos o 
soluciones especificas de una comunidad. STIX busca permitir el intercambio de 
información de forma comprensiva, rica, de alta fidelidad entre organizaciones, 
comunidades, productos y servicios ofrecidos. STIX es un lenguaje para la 
especificación, captura, caracterización y comunicación de información de 
seguridad estandar. Lo hace de forma estructurada para soportar un mejor manejo 
de la información y la aplicación de automatización. Una variedad de casos de 
uso para dicha información son realizados.
\begin{itemize}
  \item Analisis de las amenazas
  \item Especificación de patrones de indicadores para la amenaza.
  \item Manejo de las actividades de respuesta
  \item Intercambiar información
\end{itemize}

STIX provee un mecanismo común para hacer frente a estos casos de uso dando 
consistencia, eficiencia, interoperabildiad y conciencia global de la situación.

STIX provee una arquitectura unificada juntando un conjunto amplio de 
información incluyendo:
\begin{itemize}
  \item Cyber Observables
  \item Indicadores
  \item Incidentes
  \item Tacticas, tecnicas y procedimientos de los adversarios
  \item Objetivos de exploits
  \item Cursos de acción
  \item Capañas de ataques
  \item Actores de ataques
\end{itemize}

Para permitir que dicha solución sea practica para cualquier caso de uso, 
lenguajes existentes y estandarizados son utilizados.

\section{Casos de uso}
\subsection{Analisis de amenazas}
Un analista revisa información estructurada y no estructurada respecto a 
actividades de amenazas de una variedad de fuentes de entrada manuales o 
automaticas. El analista buscan entender la naturaleza de las amenazas 
relevantes, identificarlas y caracterizarlas totalmente para que todo el 
conocimiento relevante de la amenaza sea totalmente expresado y evolucionado a 
traves del tiempo. Este conocimiento relevante incluye acciones relacionadas con 
la amenaza, comportamientos, capacidades, intenciones, actores atribuidos, etc. 
Por medio del entendimiento y la caracterización, el analista puede especificar 
patrones de indicadores relevantes, sugerir acciones para las actividades de 
respuesta y compartir la información con miembros de la comunidad.

\subsection{Especificación de patrones de indicadores}
Un analista especifica patrones medibles representando las características 
observables de una amenaza junto con su contexto y metadata para ser 
interpretadas, manejadas y aplicar el patrón y sus resultados. Esto puede ser 
realizado de forma manual o con la asistencia de una herramienta automatizada.

\subsection{Manejo de las actividades de respuesta}
Los tomadores de decision y el personal de operaciones trabajan en conjunto para 
prevenir o detectar actividades amenasantes y responder a los incidentes 
detectados que sean realizados por dichas amenazas. Las acciones preventivas 
pueden mitigar vulnerabilidades, debilidades, o malas configuraciones que sean 
que sean objetivos de los exploits. Luego de la detección e investigación de 
incidentes específicos, acciones reactivas pueden ser realizadas.
\emph{Prevención de amenazas} //REPASAR ESTO
Los tomadores de decisiones evaluan acciones preventivas para amenazas 
relevantes que sean identificadas y seleccionan acciones apropiadas para su 
implementación. El personal de operaciones implementa las acciones seleccionadas 
para prevenir la ocurrencia de amenazas especificas aplicando mitigaciones 
amplias o con objetivos especificos iniciados por intepretación de los 
indicadores.
\emph{Detección de amenazas} //REPASAR ESTO
El personal de operaciones aplica mecanismos (automaticos y manuales) para 
monitorear y asistir las operaciones con la finalidad de detectar la ocurrencia 
de amenazas especificas basandose en la evidencia historica, analisis del 
contexto actual, interpretación de los indicadores que se presentan. Esta 
detección se realiza generalmente por medio de patrones de indicadores.

\emph{Respuesta a incidentes}
El personal de operaciones responde a amenazas detectadas, investiga que ha 
ocurrido o esta ocurriendo, trata de identificar y caracterizar la naturaleza de 
las amenazas y lleva a cabo la mitigación o lleva a cabo cursos de acción 
preventivos. Una vez que los efectos son entendidos, el personal de operaciones 
puede implementar mitigaciones acordes o llevar a cabo cursos de acción 
correctivos.

\subsection{Compartir información de amenazas}
Los tomadores de decisión establecen políticas respecto a que tipo de 
información será compartida, además toman decisiones respecto a con quienes se 
comparte y como debería ser manejada basandose en frameworks de confianza de 
forma de mantener niveles adecuados de consistencia, contexto y control. Esta 
politica es luego implementada para compartir los indicadores de amenazas 
apropiados y otra información de amenazas.

\section{Principios de guia}
En el enfoque dado a STIX se ha buscado implementar un conjunto de principios 
con el consenso de la comunidad. Esos principios son los siguientes:
\subsection{Expresividad}
Con el fin de soportar la diversidad de casos de uso relevantes, STIX apunta a 
proveer una cobertura expresiva en todos sus casos de uso especificos en lugar 
de dirigirse especificamente a alguno de ellos.
\subsection{Integración en lugar de duplicación}
Cuando STIX abarca conceptos de información estructurada para los cuales ya 
existen representaciones estandarizadas con el consenso adecuado y que se 
encuentran disponibles, se busca integrar estas representaciones a la 
arquitectura STIX en lugar de duplicar esta información innecesariamente.
\subsection{Flexibildiad}
Con el fin de soportar un amplio rango de casos e información variable con 
varios niveles de fidelidad, STIX esta diseñado para ofrecer tanta flexibilidad 
como sea posible. STIX se adhiere a una politica de permitir a los usuarios 
utilizar cualquier porcion de representaciones estandar que sean relevantes para 
un contexto dado y evita elementos obligatorios siempre que sea posible.

\subsection{Extensibildiad} //REVISAR
Con la finalidad de soportar un amplio rango de casos de uso con un potencial 
diferente de representación y para asegurar la facilitar el perfeccionamiento 
impulsado por la comunidad, se ha diseñado STIX para construir mecanismos de 
extensión para uso específicos, para usos localizados, para refinamientos del 
usuario y evolución y para facilidad de refinamiento y evolución centralizada.

\subsection{Automatización}
El diseño de STIX busca estructuras del contenido para que nos ea únicamente 
consumible o procesable por maquinas sino que también sea leíble por humanos. 
Esto es necesario para claridad y comprensibilidad durante las primeras etapas 
de desarrollo y adopción y para el uso sostenido en diversos ambientes,

\section{Arquitectura}

