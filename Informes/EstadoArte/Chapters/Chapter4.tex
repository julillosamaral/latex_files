% Chapter Template

\chapter{STIX} 
\label{Chapter4}
\lheader{Chapter 4. \emph{IODEF}} % Change X to a consecutive number; this is for the header on each page - perhaps a shortened title

%----------------------------------------------------------------------------------------
%	SECTION 1
%----------------------------------------------------------------------------------------

\section{IODEF}

Las organizaciones deben colaborar entre ellas para mitigar las actividades 
maliciosas que atacan sus redes así como para ganar conocimiento de posibles 
amenazas. Esta coordinación puede requerir coordinación con ISPs, sitios remotos 
o intercambiar datos con socios. 
IODEF son las siglas para Incident Object Description Exchange Format, este 
define una representación de datos que provee un framework para el intercambio 
de información entre CSIRTs, dicho tipo de información es intercambiado 
comúnmente por CSIRTs y es referente a incidentes de seguridad. IODEF provee una 
representación en XML para transportar información de incidentes entre dominios 
administrativos entre pares que tienen una responsabilidad operacional de 
remediar o de analizar y establecer advertencias en un dominio definido. El 
modelo de datos provisto en IODEF codifica la información referente a hosts, 
redes y servicios corriendo en estos sistemas; metedologia de ataques y 
evidencia forense asociada; impacto de la actividad realizada; aproximaciones 
limitadas para documentar el flujo.
El objetivo primordial de IODEF es mejorar las capacidades operacionales de los 
CSIRTs. La adopción por parte de la comunidad provee una habilidad mejorada para 
resolver los incidentes y transmitir el contexto  simplificando la colaboración 
e intercambio de datos. El formato estructurado provisto por IODEF permite:
\begin{itemize}
  \item Incrementar la automatización en el procesamiento de los datos, ya que 
  los recursos de los analistas de seguridad de parsear documentos se verá 
  reducido.
  \item Bajar el esfuerzo necesario para normalizar datos similares de 
  diferentes fuentes.
  \item Un formato común con el cual construir herramientas interoperables para 
  el manejo de incidentes y análisis subsecuente, específicamente cuando los 
  datos provienen de dominios distintos.
\end{itemize}
La coordinación entre CSIRTs no es un problema estrictamente técnico. Hay muchas 
consideraciones: procedimientos, confianza, legales, las cuales pueden ocasionar 
que las organizaciones intercambien información. IODEF no busca evadir dichas 
consideraciones, sin embargo, las implementaciones operacionales de IODEF deben 
considerar este contexto.

\subsection{Modelo de datos de IODEF}
A la hora de diseñar IODEF se realizaron ciertas consideraciones de diseño
\begin{itemize}
  \item El modelo de datos sirve como formato de transporte. Por ello, su 
  representación especifica no es optima para almacenamiento en disco, 
  archivamiento a largo plazo o procesamiento en memoria.
  \item Por medio de la implementación no se busca establecer un consenso 
  respecto a la definición de un incidente. Se busca en su lugar un 
  entendimiento amplio que sea lo suficientemente flexible como para abarcar la 
  mayoría de las operaciones.
  \item Describir un incidente para todas las definiciones requeriría un modelo 
  de datos extremadamente complejo. Por ello, IODEF solo busca dar un marco para 
  transmitir información de incidentes comúnmente intercambiada. Se asegura un 
  mecanismo amplio para ser extendido para soportar información propia de la 
  organización, y técnicas para referenciar información mantenida por fuera del 
  modelo de datos.
  \item El dominio del análisis de seguridad no esta totalmente estandarizado y 
  debe basarse en descripciones textuales libres. IODEF busca conseguir un 
  balance entre el contenido libre, pero permitiendo el procesamiento automático 
  de la información de los incidentes.
  \item IODEF es solo una de las representaciones que han sido estandarizadas. 
  El modelo de datos de IDMEF influencio el diseño de IODEF.
\end{itemize}

\subsection{Implementaciones de IODEF}
Las implementaciones del protocolo se especifican como un esquema XML. 
Implementar IODEF en XML provee varias ventajas. Que sea extensible lo hace 
ideal para especificar codificación de datos que soporta varias codificaciones 
de caracteres. Es mas simple la manipulación debido a la presencia de varias 
tecnologías para ello. Aunque fundamentalmente XML es una representación de 
texto, lo cual lo hace ineficiente cuando se deben embeber datos binarios o 
grandes volúmenes de datos deben ser intercambiados.

\subsection{Internacionalización}
Internacionalización y localización son de interés para IODEF, dado que solo por 
medio de colaboración (a menudo con barreras idiomáticas) son resueltos ciertos 
incidentes. IODEF soporta esto dependiendo de las construcciones de XML, y por 
medio de diseño explícito en el modelo de datos.
Como IODEF es implementado como un esquema XML, este soporta las diferentes 
codificaciones de caracteres. Además, los documentos IODEF deben especificar el 
lenguaje en el cual sus contenidos son codificados. 

\subsection{Consideraciones de Seguridad}
El modelo de datos de IODEF no introduce problemas de seguridad. Este solo 
define una representación simple para información de incidentes. Como los datos 
codificados por IODEF pueden ser considerados sensibles por las partes que los 
intercambian o por los descritos por los datos, se deben tomar precauciones para 
asegurar la confidencialidad durante el intercambio y el subsecuente 
procesamiento. El primero debe ser resguardado por un formato de mensaje, pero 
luego se deben tener consideraciones de seguridad por el sistema que procesa los 
datos, los almacena y archiva la documentación y la información derivada de 
estos.
El contenido de un documento IODEF puede incluir un pedido de acción o un parser 
puede independientemente tener lógica para realizar cierta acción basandose en 
información que encuentra. Por esta razón, se debe tener cuidado de autenticar 
apropiadamente el beneficiario del documento y atribuir un nivel de 
confidencialidad apropiado a los datos antes de realizar la acción.
El formato de mensaje subyacente  y el protocolo utilizado para intercambiar 
datos provee una garantía de confidencialidad, integridad  y autenticidad. El 
uso de protocolos de seguridad estandarizados es recomendado. Por ejemplo el uso 
de IODEF/RID.
Con el fin de sugerir buenas practicas en el  procesamiento y manejo de los 
datos codificados, IODEF  permite a un emisor de documentos transmitir una 
política de privacidad utilizando un atributo de restricción. Las distintas 
instancias de este atributo permite a los distintos elementos del documento 
tener las mismas políticas. Esto sirve como una guía para el receptor, pero este 
podría decidir ignorarlo, este problema no es un reto técnico.

\section{IDMEF}
IDMEF es un protocolo experimental que no especifica ningún estándar. Sin 
embargo IODEF se basa en algunas de las cosas definidas por este para su 
definición. El propósito de IDMEF es definir formatos de datos y procedimientos 
de intercambio para compartir información de interés con sistemas de detección 
de intrusos y sistemas de respuesta con los sistemas de administración que 
podrían necesitar interactuar con estos.
El rfc de IDMEF describe el modelo de datos para representar información tomada 
de los sistemas de intrusión y explica la razón de usar este modelo.
IDMEF busca ser un formato de datos estándar el cual pueda ser utilizado por los 
IDSs para reportar alertas sobre eventos que parecen sospechosos. Se puede 
proveer interoperabildiad entre sistemas comerciales, open source y sistemas de 
investigación, permitir a los usuarios esta mezcla de sistemas ayuda a obtener 
una implementación óptima de sus sistemas.

\subsection{Modelo de datos de IDMEF}
El modelo de datos de IDMEF es una representación orientada a objetos de los 
datos del alerta enviados a los administradores de los sistemas de intrusión.
El modelo de datos presenta varios problemas referentes a la representación de 
los datos:
\begin{itemize}
  \item La información de alertas es mayormente heterogénea. Algunas alertas son 
  definidas con muy poca información, como orígenes, destinos, nombre u hora del 
  evento. Otras alertas proveen mucha mas información, como puertos de 
  servicios, procesos, información de usuarios, etc. El modelo de datos que debe 
  representar dicha información debe ser flexible para adaptarse a las distintas 
  necesidades. Un modelo orientado a objetos es extensible por medio de 
  agregación y sub clases. Si una implementación del modelo de datos extiende 
  con una nueva clase, por medio de agregación o subclases, una implementación 
  que no entienda estas extensiones podrá seguir entendiendo el subconjunto de 
  información que esta definido en el modelo. Estas dos formas de extender el 
  modelo permiten que se mantenga la consistencia del modelo.
  \item Los IDS son diferentes, algunos analizadores detectan ataques analizando 
  el trafico en la red, otros utilizando los logs de sistemas operativos o 
  aplicaciones que auditan información. Alertas para el mismo ataque enviadas 
  por analizadores con diferentes fuentes de información, no contendran los 
  mismos datos.
  El modelo de datos define clases que soportan las diferencias en las fuentes 
  de los datos. En particular, las nociones de fuente y objetivo para el alerta 
  son representadas por una combinación de nodo, procesos, servicio y clase de 
  usuario.
  \item Las capacidades de los analizadores son diferentes. Por ello el modelo 
  de datos debe permitir la conversión de formatos utilizados por herramientas 
  distintas a IDSs con el propósito de mayor procesamiento de la información.
  \item Los ambientes operacionales son diferentes. Dependiendo en el tipo de 
  red o sistemas operativos utilizados, los ataques serán observados y 
  reportados con diferentes características. El modelo de datos se 
  adapta a estas diferencias.
  \item Los instrumentos comerciales persiguen objetivos diferentes. Por varias 
  razones, estos desean entregar mas o menos información sobre ciertos tipos de 
  ataques. El modelo orientado a objetos permite la flexibilidad necesaria 
  preservando la integridad del modelo.
\end{itemize}

\subsection{El diseño del modelo de datos}
Representación de eventos
El objetivo del modelo de datos es proveer una representación estándar de la 
información que el analizador de un sistema de intrusión reporta cuando detecta 
la ocurrencia de eventos inusuales. Estas alertas pueden ser simples o complejas 
dependiendo de las capacidades del analizador que la creo.

