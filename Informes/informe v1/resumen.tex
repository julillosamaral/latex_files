\begin{abstract}
\label{abstract}
En los últimos tiempos la sofisticación, velocidad, y el impacto de los ataques referidos a seguridad informática ha crecido. A su vez  los actores que realizan esos ataques  son cada vez más hábiles y persistentes en busca de información específica, como ser credenciales de acceso, información personal, tarjetas de crédito, espionaje industrial, o incluso información de seguridad nacional. En este contexto es necesario contar con  estrategias de defensa, las cuales a su vez necesitan  adaptarse constantemente a los nuevos actores y ataques.\\

El intercambio de información ha demostrado ser fundamental en la lucha contra las nuevas modalidades de ataque. Hoy en día, un número cada vez mayor de organizaciones comparten  información sobre amenazas para obtener una visión más completa de la actividad del atacante y ayudar a priorizar las defensas propias de la organización.\\

Existen varias iniciativas que apuntan a automatizar dicho intercambio de información, en particular MITRE [REF] ha propuesto el lenguaje denominado STIX con el objetivo de contar con un estándar que pueda ser utilizado para la normalización y representación  de amenazas en forma estructurada. El lenguaje STIX provee construcciones que permiten representar  toda la gama de información potencial relacionada con amenazas cibernéticas y se ha puesto especial esfuerzo en lograr que el mismo sea totalmente expresivo, flexible, extensible, automatizable, y tan legible como sea posible.\\

Relacionado al lenguaje STIX se encuentra \textit{Trusted Automated eXchange of Indicator Information} (TAXII), que es el mecanismo principal de transporte para la información representada en dicho lenguaje. Según expresa el propio proyecto de MITRE “Mediante el uso de TAXII, las organizaciones pueden compartir información de amenaza de una manera segura y automática.”\\

Por otro lado tenemos sistemas de seguimiento de incidentes, por ejemplo  el sistema llamado \textit{Request Tracker  for Incident Response} (RTIR). RTIR es un proyecto de código abierto y es un  sistema de gestión dirigido específicamente a  equipos de respuesta de seguridad informática (CSIRT).  Un flujo de trabajo típico de un CSIRT comienza por realizar actividades de triage (clasificación) sobre los informes de incidentes entrantes y vincularlos a un incidente existente o crear uno nuevo. A cada incidente se le realiza un seguimiento, donde se manifiesta todo lo que se necesita saber para resolver el problema. También se ponen en marcha investigaciones para trabajar con otras organizaciones,como ser proveedores, otros equipos de respuesta, consultores, etc.\\

A la fecha, en las versiones actuales del sistema se proveen funcionalidades para realizar este tipo de tareas, pero el workflow de trabajo subyacente no provee mecanismos para el intercambio de información, mucho menos utilizando algún tipo de estándar .\\

En este contexto es que el equipo de respuesta a incidentes de Tilsor (CSIRT Tilsor) en conjunto con el Grupo de Seguridad Informática de la Facultad de Ingeniería, proponen el presente proyecto de grado. El proyecto pretende dotar al sistema RTIR, u otro sistema similar, con los mecanismos necesarios para el intercambio de información mediante el uso de los lenguajes STIX y el mecanismo de transporte TAXII mencionado anteriormente.
\end{abstract}