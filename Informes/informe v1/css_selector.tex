
\chapter{Selectores de CSS}
Entender como los navegadores parsean las reglas de estilo y renderizan las páginas web es una tarea importante al momento de mejorar la performance de una página web.
A medida que los navegadores parsean el documento HTML, construyen una estructura arborescente que representa todos los elementos que van a ser desplegados en la página.
Una vez completa esta etapa, busca las reglas que aplican a los elementos en diversas hojas de estilo, de acuerdo a las reglas de cascada, herencia y orden de CSS.

En la mayoría de las implementaciones de los navegadores, el motor de CSS busca reglas en las hojas de estilo que aplican a cada elemento del documento, evaluándolas de derecha a izquierda 
(debido a que el parser que utilizan es \emph{bottom-up}), empezando por el selector de más a la derecha, conocido como \emph{key} y
moviendose a través del selector hasta encontrar una concordancia o descartando la regla en otro caso.

En base a este sistema, mientras menos reglas tengan que ser evaludas por el motor de estilos, mejor será la performance del mismo. Por lo tanto remover reglas que no aplican es
un paso importante para mejorar la performance del renderizado, pero aún más importante es optimizar las definiciones de las reglas. Un aspecto clave al optimizar
las reglas de estilo, es definir reglas para que sean lo mas específicas posible y evitar redundancia en las mismas, para que de esta forma el motor de estilos pueda
encontrar rápidamente concordancias, evitando perder tiempo evaluando reglas que no aplican.

\section{Tipos de selectores}
En esta sección se listan los distintos tipos de selectores de CSS ordenados en base al costo en cuanto a la performance de los mismos, comenzando por los que tienen
menor impacto en la performance al momento de aplicar las reglas.

\subsection{Id}
Simple y eficiente, este selector aplica al único elemento en el documento HTML cuyo atributo \emph{ID} coincida con el de la regla.

\subsection{Clase}
Las reglas basadas en clases, se especifican con un punto seguido por el nombre de la clase. Los selectores de clase se aplican a todos los elementos que contengan la clase
de la regla en su atributo clase.

\subsection{Tipo}
Los selectores de tipo aplican a todos los elementos de un tipo especifico. La regla \emph{a\{text-decoration: none;\}} remueve el subrayado del texto de todos los
\emph{anchors} que se encuentren en la página. El uso de estas reglases una forma eficiente de agregar estilos a todos los elementos de un tipo especifico sin tener que
agregar caracteres extra, como clases a los elementos.

\subsection{Hermano adyacente}
Este selector consiste en concatenar dos selectores mediante el simbolo +. La regla \emph{h1 + \#toc} aplica al elemento que tiene \emph{toc} como atributo \emph{ID}, cuyo
hermano previo es un elemento del tipo \emph{h1}.

\subsection{Hijo \emph{child}}
Este tipo de selector es formado por la combinacion de dos o mas selectores simples mediante el uso del símbolo \emph{\textgreater}. La regla \emph{\#toc \textgreater   li \{font-weight: bold;\}}
aplica a todos los elementos del tipo \emph{li}, cuyo padre es el elemento con el atributo \emph{ID} igual a toc.

\subsection{Descendiente}
Estos selectores utilizan un espacio \emph{' '} como combinador. La siguiente regla \emph{\#toc a \{ color: \#444\}}, aplica a todos los elementos del tipo \emph{a} que sean
descendientes de el elemento cuyo \emph{ID} es igual a toc.

\subsection{Universales}
Los selectores universales (reglas representadas con un *) aplican a todos los elementos del documento.

\subsection{Atributo}
Los selectores por atributo aplican basandose en la existencia o valor de los atributos de un elemento. Un ejemplo de esto es la siguiente regla
\emph{[href='\#index'] \{font-style: italic;\}}.

\subsection{Pseudo-Clase y Pseudo-Elementos}
Este tipo de selectores son utilizados para aplicar a situaciones en las cuales la información no esta representada en el DOM. Algunas de las \emph{pseudo-classes} más
utilizadas son :hover, :visited, :first-child, :focus, etc.

\section{Reglas a evitar}

\subsection{No sobrecargar los selectores por Id}
Debido a que existe un único elemento en la página con un id específico, no es necesario agregar clasificadores extra ya que la única consecuencia que tiene esto es
retrasar el proceso. Por lo que es recomendable evitar reglas del estilo \emph{div\#toc}.

\subsection{No sobrecargar selectores de clase}
En vez de clasificar selectores de clase para etiquetas específicas, se recomienda extender el nombre de la clase para que sea específico con el caso de uso que se quiere
tratar. Por ejemplo, si se tiene una regla de este tipo \emph{li.chapter}, se recomienda cambiarla por \emph{.li-chapter} o mejor aun \emph{.list-chapter}.

\subsection{Reglas especificas}
La principal causa de la disminución de la eficiencia de la aplicación de los selectores es la existencia de múltiples etiquetas en una regla. Es mejor dividir estos
selectores en clases y agregarlas a los elementos correspondientes. Por ejemplo, es recomendable cambiar las reglas del tipo \emph{ol li a} por selectores de clases
del tipo \emph{.list-anchor}.

\subsection{Evitar selectores de descendencia y \emph{child}}
Los selectores por descendencia son los más costosos de procesar, debido a que por cada elemento que aplica a cada parte de la regla se debe evaluar otro nodo más. En
vez de estos es mejor utilizar clases asociadas a los elementos de la regla.

\subsection{Utilizar herencia de los atributos}
Muchas de las reglas de CSS se heredan, es mucho más costoso especificar este tipo de propiedades en cada elemento a tenerlas en un ancestro del mismo del cual
pueden heredarlas.
