{\huge \bfseries Resumen \\[0.4cm]}


Este proyecto tiene como objetivo específico el desarrollo de herramientas para el intercambio seguro de información sensible entre organizaciones. El cibercrimen es una realidad a la que no se le puede dar la espalda, y en estos últimos años organizaciones gubernamentales, corporaciones, empresas y otro tipo de instituciones han visto como sus activos críticos son puestos en riesgo como resultado de ataques perpetrados en forma sistemática, usualmente a través de la Internet. \\

La comunidad de Seguridad Informática ha identificado el intercambio entre organizaciones de información relativa a incidentes de seguridad como una herramienta más para la defensa ante ese tipo de ataques. En particular, cuando se trata de equipos de respuesta a incidentes de seguridad, se considera esencial poder propiciar e implementar este intercambio de información.
Existen varias iniciativas que apuntan a  proveer un marco formal para dar soporte a la automatización de intercambio de información sensible, en particular MITRE \cite{mitre} ha propuesto el lenguaje denominado STIX con el objetivo de contar con un estándar que pueda ser utilizado para la representacion de amenazas en forma estructurada y el protocolo TAXII que define un conjunto de servicios y procesos para el intercambio de información. \\

El resultado principal de este proyecto lo constituye el diseño e implementación de un prototipo de una herramienta para el intercambio de información sensible entre organizaciones. La misma incluye una implementación básica del protocolo TAXII que se integra al conjunto de funcionalidades provistas por el sistema \textit{Request Tracker for Incident Response} (RTIR), que implementa servicios de gestión de tickets, y que es una extensión de \textit{Request Tracker} RT orientada a proveer funcionalidades específicas para la gestión de incidentes que debe realizar un equipo de respuestas a incidentes de seguridad (CSIRT). La información que la herramienta permite  intercambiar está estructurada utilizando conceptos primitivos básicos del lenguaje STIX, como lo es la noción de Cyber-observable.
