{\huge \bfseries Resumen \\[0.4cm]}

Durante el estado del arte se estudió que el intercambio de información da a las organizaciones una ventaja respecto a sus atacantes. Se vio que dicho intercambio ayuda a mejorar la asignación de los recursos de forma de obtener el mejor resultado posible de sus defensas. De dicho estudio surgieron los requerimientos buscados en una herramienta de intercambio de información de seguridad. \\

Existen varias iniciativas que apuntan a automatizar dicho intercambio de información, en particular MITRE \cite{mitre} ha propuesto el lenguaje denominado STIX con el objetivo de contar con un estándar que pueda ser utilizado para la normalización y representación  de amenazas en forma estructurada. El lenguaje STIX provee construcciones que permiten representar  toda la gama de información potencial relacionada con amenazas cibernéticas y se ha puesto especial esfuerzo en lograr que el mismo sea totalmente expresivo, flexible, extensible, automatizable, y tan legible como sea posible.\\
 
Durante el proyecto se utilizó el protocolo TAXII para intercambiar información de seguridad, dicha información es representada por medio del lenguaje STIX mencionado anteriormente. La herramienta diseñada integra STIX y TAXII con \textit{Request Tracker  for Incident Response} (RTIR) proporcionandole una representación estructurada de la información y un método de intercambio de información de seguridad.

