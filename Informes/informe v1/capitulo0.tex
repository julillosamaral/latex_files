{\huge \bfseries Resumen \\[0.4cm]}

Durante el estado del arte se aprendió que el intercambio de información da a las organizaciones una ventaja respecto a sus atacantes. Se vio que dicho intercambio ayuda a mejorar la asignación de los recursos de forma de obtener el mejor resultado posible de sus defensas. De dicho estudio surgieron los requerimientos buscados en una herramienta de intercambio de información de seguridad. \\

Existen varias iniciativas que apuntan a automatizar dicho intercambio de información, en particular MITRE \cite{mitre} ha propuesto el lenguaje denominado STIX con el objetivo de contar con un estándar que pueda ser utilizado para la representación  de amenazas en forma estructurada. El lenguaje STIX busca ser expresivo, flexible, extensible, automatizable y tan legible como sea posible. Junto con el protocolo TAXII, también desarrollado por MITRE buscan ayudar a las organizaciones a lograr un mejor intercambio de información.

\textit{Request Tracker for Incident Response} (RTIR) es una herramienta muy utilizada por los CSIRTs para la gestión de incidentes. Esta no cuenta con una representación estructurada de la información ni con un mecanismo de intercambio de información con otras organizaciones. Por ello se realizó una implementación utilizando las iniciativas de MITRE antes mencionadas que le proveyera dichas caracteristicas.

