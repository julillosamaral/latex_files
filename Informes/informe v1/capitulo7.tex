\chapter{Conclusiones y trabajo futuro}
\label{capitulo7}

En este capítulo se presentan las principales conclusiones y dificultades más relevantes del proyecto. Por último, se proponen posibles líneas de trabajo a futuro.

\section{Conclusiones}

En los últimos años se ha visto un incremento en la cantidad de equipos conectados a las redes de las organizaciones, a su vez han aumentado los usuarios malintencionados. Estos cuentan cada vez con mejores herramientas y procedimientos mas sofisticados con los cuales realizar ataques. Debido a esto es necesario que las estrategias de defensa se adapten a los nuevos actores y ataques. Para responder a esto, las organizaciones han recurrido al intercambio de información de amenazas con la finalidad de identificar de mejor forma las actividades de sus adversarios y asi obtener los mejores resultados posibles de sus defensas.\\
\bigskip
Durante el estado del arte se vieron diferentes problemas que dificultan el intercambio de información de seguridad entre las organizaciones. Estos van desde problemas que son estrictamente técnicos a otros que son de índole política. En este punto se decidió centrarse en los problemas que son estrictamente técnicos y se identificaron cuales son los enfoques utilizados en la actualidad para el intercambio de información. Se vio que las  herramientas que son utilizadas actualmente se enfocan en una parte del problema y que en muchos casos el intercambio de información es realizado entre humanos.  \\
\bigskip
Durante el transcurso del proyecto se vio el fuerte interés que existe en las organizaciones por llegar a un consenso respecto a estándares utilizados en el intercambio de información de seguridad. En particular resultaron llamativas las iniciativas de MITRE que cuentan con el apoyo tanto de organizaciones provenientes del sector publico así como del sector privado y las cuales provienen de distintos sectores de la industria.\\   
\bigskip
La existencia de estos estándares es fundamental para lograr un buen nivel de automatización. Si bien se puede automatizar sin que un estándar sea apoyado por la comunidad, el esfuerzo que se debe realizar para poder integrar conocimientos es muy grande. Esta integración de conocimientos es clave para fomentar el avance de las herramientas, técnicas y metodologías, que permitan contrarrestar el impacto de los ataques automatizados.\\
\bigskip
El esfuerzo encabezado por MITRE busca desarrollar un conjunto de estándares referentes a la caracterización e intercambio de información de seguridad. Intenta proveer lenguajes que permitan facilitar la comunicación de la información de seguridad de forma precisa. Algunos de los lenguajes son CYBOX, STIX, CAPEC, entre otros. Junto con estos esfuerzos utilizados para la representación de la información se encuentra TAXII utilizado para el intercambio de información.\\
\bigskip
También durante el estudio del arte se estudió RTIR y algunas otras herramientas que cuentan con características similares. RTIR ha sido una herramienta desarrollada junto con CSIRTs para el manejo de incidentes de seguridad. Permitiendo que se agregue información asociada a estos. En el transcurso del proyecto se tuvo que adquirir un manejo de la herramienta para realizar el trabajo.\\
Durante la implementación de la herramienta se desarrolló un plugin para RTIR, el desarrollo de dicho plugin no fue una tarea sencilla. Se desarrolló por medio de ingeniería reversa de otro plugin debido a que no se encontró información precisa respecto a como realizar la implementación de un nuevo plugin. Esto sucedió a pesar de que RTIR tiene una muy buena documentación.
\bigskip
El estudio del arte no fue una tarea sencilla debido a la amplitud de la temática tratada y por la cantidad de información encontrada. Se tuvo a disposición mucha información de múltiples fuentes. El trabajo se centro principalmente en la representación estrucutrada de la información y su intercambio entre organizaciones.\\
\bigskip
La herramienta requirió el desarrollo por medio de componentes que se integraron utilizando de una API REST. La primera fue TAXII App que se desarrolló por medio del framework Django y la otra fue el plugin RTIR que se desarrolló con scripts en lenguaje PERL.
Una lección que se aprendió de esto son las bondades que nos ofrecen los frameworks para realizar el desarrollo. El producto realizado por medio del framework Django es de una calidad muy superior al plugin RTIR implementado. En la mejora de la implementación de TAXII App tuvo un fuerte impaco que Django es un framework que trabaja con el patron MVC, antes de realizar el proyecto ya se contaba con experiencia utilizando dicho patrón. Además Djanco cuenta con una comunidad muy grande de usuarios lo cual facilita la busqueda de información y el aprendizaje.\\
\bigskip
Además se aprendió a trabajar solo en un proyecto extenso, siguiendo un procedimienta establecido, donde primero se hizo un estudio del estado del arte, luego se ideó la solución, para luego implementarla.\\
\bigskip
Se logró diseñar e implementar una herramienta utilizando STIX, TAXII y RTIR que permite el intercambio y representación de información de seguridad de forma estructurada. Además dicha herramienta resuelve algunos de los requerimientos planteados durante el análisis. Dichos requerimientos fueron obtenidos a partir de algunas de las problematicas del intercambio de información de seguridad, dichas problematicas fueron encontradas durante el estado del arte.\\
Se pudo diseñar una arquitectura extensible y modular, que permite permita la interacción con herramientas de gestión de incidentes distintas a RTIR y dandole nuevas funcionalidades al sistema ya existente.\\
Fue fundamental durante el desarrollo el realizar un prototipo para validar la arquitectura planteada. Dicho prototipo fue utilizado luego como base para el desarrollo de la herramienta.

\section{Trabajo Futuro}
Durante el transcurso del proyecto se identificaron distintas líneas de trabajo futuro sobre las cuales se puede trabajar para mejorar la herramienta. A continuación delineamos algunos de los trabajos que se podrían seguir realizando en el marco de este proyecto.
\bigskip

Como es posible que se intercambien grandes volúmenes de datos entre organizaciones, es necesario que se diseñe e implemente un método para correlacionar la información intercambiada, de esta forma se agrupan los datos, facilitando de esta forma el trabajo de los analistas. Con dicha información correlacionada se pueden presentar reportes a los analistas resumiendo el contenido existente en el sistema. Durante el diseño y análisis de la herramienta se tuvo en consideración dicha funcionalidad pero no se implementó y no hubo una investigación profunda referente a esto, por lo cual se deja para un trabajo a futuro. 
De la misma forma se consideró durante el diseño y análisis la sanitización de la información pero tampoco se implementó en el prototipo dicha funcionalidad quedando para un trabajo a futuro.  Dicha funcionalidad permite filtrar datos sensibles que podrían comprometer la imagen o la integridad de las organizaciones.
\bigskip

Respecto a la implementación realizada, queda planteado realizar una mejor interacción entre RTIR y el \textit{plugin} de forma de extender las funcionalidades ya existentes en RTIR así como agregar algunas nuevas. Entre dichas mejoras se encuentra la mejora de la experiencia de usuario al usar RTIR junto con TAXII App, sería deseable que al momento de que un analista cree un ticket dentro de RTIR, este ya pueda ser asociado a información perteneciente a TAXII App. Además es deseado permitir agregar nuevos cyber observables a incidentes ya existentes. Esto, facilitaría el trabajo de los analistas.

\bigskip

Otro de los puntos a mejorar en la implementación del cliente TAXII es el seguimiento de los usuarios u organizaciones con las que se intercambia la información de seguridad y filtrar a que datos pueden acceder los distintos usuarios. De esta forma se podría realizar un modelo RBAC para el acceso a la información.
\bigskip

Sería interesante probar el sistema en un ambiente real, con grandes cantidades de datos.
