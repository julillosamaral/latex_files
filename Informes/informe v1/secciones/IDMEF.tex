\subsubsection{IDMEF}

\textit{Intrusion Detection Message Exchange Format} (IDMEF) fue especificado como un 
protocolo experimental que no especifica ningún estándar. El propósito de 
IDMEF es definir formatos de datos y procedimientos de intercambio para 
compartir información de interés con sistemas de detección de intrusos y 
sistemas de respuesta con los sistemas de administración que deben 
interactuar con estos. El RFC de IDMEF describe el modelo de datos para 
representar información tomada de los sistemas de intrusión. Se busca que dicho 
formato pueda ser utilizado por los \textit{Intrusion Detection Systems} (IDSs) para 
reportar alertas sobre eventos que parezcan sospechosos.\\

\paragraph{Modelo de datos de IDMEF}\ \\
El modelo de datos de IDMEF es una representación orientada a alertas enviadas a 
los administradores desde los sistemas de intrusión. Su diseño busca proveer una representación estándar de las 
alertas de manera que la información no sea ambigua y que se describa la 
relación entre alertas simples y complejas.\\

IDMEF tiene como objetivo proveer una representación estándar de la 
información analizada por un sistema de intrusión cuando es detectado 
un evento inusual. Estas alertas pueden ser simples o complejas 
dependiendo de las capacidades de la herramienta que las creo.\\

Por medio de IDMEF se pueden referencias contenidos adicionales.
es importante debido a que la tarea de clasificar y nombrar vulnerabilidades es 
difícil y sumamente subjetiva. El modelo de datos no debe ser ambiguo, por lo 
que mientras que se permite que los análisis sean más o menos precisos 
entre ellos, no se debe permitir que estos produzcan información contradictoria  
en dos alertas que describen el mismo evento. De todas formas, siempre es 
posible insertar toda la información útil de un evento en campos de extensión 
de la alerta en lugar de en los campos a los que estos pertenecen, sin embargo, 
dichas prácticas reducen la interoperabilidad y deberían ser evitadas en lo 
posible.


