\subsubsection{RID}

\textit{Real-Time Inter-Network} (RID) es un método de comunicación entre redes para 
facilitar el intercambio de datos de incidentes. A su vez, busca integrar 
mecanismos existentes de detección, seguimiento, identificación de fuentes y 
mitigación que aporta una solución al manejo de incidentes. Combinar estas 
capacidades en un sistema de comunicación permite incrementar el nivel de 
seguridad en la red. Las políticas para manejar incidentes son recomendadas y 
pueden ser acordadas por un consorcio utilizando recomendaciones y 
consideraciones de seguridad.\\

RID ha sido ampliamente utilizando en comunidades de investigación, pero no ha 
sido muy adoptado por otros sectores. Fue desarrollado como un mecanismo de 
comunicación para facilitar la transferencia de información entre distintos 
proveedores de servicios de Internet para trazar precisa y eficientemente el 
flujo de paquetes nocivos a lo largo de la red.\\

Los datos en RID son representados como documentos XML utilizando IODEF. De esta 
forma se simplifica la integración con otros aspectos del manejo de incidentes.\\ 

Se deben utilizar 
conexiones autenticadas y encriptadas entre sistemas RID para proveer 
confidencialidad, integridad, autenticidad y privacidad de los datos. \\

RID requiere una 
especificación de un protocolo de transporte para asegurar la interoperabilidad 
entre las organizaciones socias. El RFC 6046, especifica 
el transporte de mensajes RID sobre HTTPS/TLS. 












