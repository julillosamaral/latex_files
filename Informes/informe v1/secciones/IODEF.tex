\subsubsection{IODEF}

\textit{Incident Object Description Exchange Format} (IODEF) define una representación de datos que provee un framework para el 
intercambio de información entre CSIRTs, dicho tipo de información de seguridad 
es comúnmente intercambiado entre este tipo de organizaciones. IODEF provee una 
representación en XML para transportar información de incidentes entre pares. El modelo de datos provisto por IODEF codifica la información 
referente a hosts, redes y servicios corriendo en estos sistemas; metodología 
de ataques y evidencia forense asociada; impacto de la actividad realizada y
aproximaciones a los trabajos realizados.\\

IODEF es compatible y tiene la capacidad de incluir mensajes 
IDMEF. Los actores principales en IODEF son los CSIRTs, por lo tanto se busca que sea utilizado por personas. Esto causa que IODEF deba ser leíble por humanos además de parseable por computadoras.\\

Los mensajes IODEF consideran 
información referente al manejo de incidentes, almacenamiento de dicha 
información o estadísticas y análisis.\\

El objetivo primordial de IODEF es mejorar las capacidades operacionales de los 
CSIRTs. La adopción por parte de la comunidad provee una habilidad mejorada para 
resolver los incidentes y transmitir información de contexto simplificando la 
colaboración e intercambio de datos.
El formato estructurado provisto por IODEF permite:
\begin{itemize}
  \item Incrementar la automatización en el procesamiento de los datos.
  \item Bajar el esfuerzo necesario para normalizar datos similares de 
  diferentes fuentes.
  \item Un formato común con el cual construir herramientas interoperables para 
  el manejo de incidentes y análisis subsecuente, específicamente cuando 
  los datos provienen de dominios distintos.
\end{itemize}

La coordinación entre CSIRTs no es un problema estrictamente técnico. La 
confianza, los procedimientos y las leyes son consideraciones que pueden impedir 
que las organizaciones intercambien información. IODEF no busca evadir dichas 
consideraciones, sin embargo, las implementaciones operacionales de IODEF deben 
considerar este contexto.