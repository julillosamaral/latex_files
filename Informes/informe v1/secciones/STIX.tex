\subsection{STIX}

Structured Threat Information eXpression (STIX) es un lenguaje para la especificación, captura, representación y 
comunicación de información de seguridad. Lo realiza de manera 
estructurada para soportar un mejor manejo de la información así como para 
permitir procesos automatizados.\\

Se busca que la información sea representada de forma estructurada, siendo dicha información expresiva, flexible, extensible, automatizable y legible. La información debe ser estructurada para que pueda ser parseada por una máquina pero no se debe perder la posibilidad de que sea leeida y entendida por un analista. \\

Al convertirse STIX en un estándar para la comunidad se obtendrá información de 
mejor calidad la que será mejor aprovechada por las organizaciones.

A diferencia de IODEF, STIX provee una lista de elementos para construir 
indicadores de compromiso, esto da la posibilidad de representar al conjunto de artefactos, técnicas y procedimientos descubiertos en un análisis forense, dichos elementos provienen de una intrusión y pueden ser identificados en un dispositivo o red. Además, STIX se integra con otras iniciativas de MITRE como CyBox \cite{cybox}, CAPEC \cite{capec}, MAEC \cite{maec} o CVE \cite{cve}. Además de dar soporte a la representación de tácticas, técnicas y procedimientos del adversario también permite representar tareas 
realizadas por la organización. IODEF fue pensado para compartir información de 
incidentes y no indicadores de compromiso. STIX permite el intercambio de 
indicadores referentes a amenazas así como información del contexto en el que 
éstas se dan. Juntos, permiten tener un conocimiento más rico de las 
intenciones, capacidades, motivaciones y actividades de un adversario y de esta 
forma defenderse de él.\\

STIX plantea una mejora respecto a los enfoques que se tienen actualmente en los que la información que se intercambia y utiliza es atómica, inconsistente 
y muy limitada en sofisticación y expresividad. En dichos enfoques se utilizan estructuras que se enfocan en una parte del problema. %Además, el uso de estructuras 
%estandarizadas se enfoca en una porción del problema. %y la integración no se 
%realiza de forma adecuada entre si o carece de la flexibilidad para hacerlo. 
Generalmente, las actividades de intercambio de indicadores son entre humanos, 
siendo los indicadores desestructurados o semi-estructurados e intercambiados 
por medio de portales web o encriptados y enviados vía email. Recientemente se 
ha visto el surgimiento de transferencias entre máquinas, éstas intercambian 
conjuntos de indicadores simples de modelos de ataques bien conocidos.\\


STIX busca extender los indicadores para permitir el manejo e intercambio de 
estos de forma más expresiva y con un espectro más amplio de información.\\

STIX provee una arquitectura unificada con un conjunto amplio de información 
para permitir una solución práctica que permita la implementación de distintos 
casos de uso. El conjunto de información incluye:
\begin{itemize}
  \item Cyber Observables
  \item Indicadores
  \item Incidentes
  \item Tácticas, técnicas y procedimientos de los adversarios
  \item Objetivos de exploits
  \item Cursos de acción
  \item Campañas de ataques
  \item Actores de ataques
\end{itemize}

\subsubsection{Principios de desarrollo de STIX}\ \\
Con el consenso de la comunidad, se han definido un conjunto de principios para el desarrollo de STIX. Estos principios son los siguientes:
\begin{itemize}
  \item \underline{Expresividad}: Con el fin de soportar la diversidad de casos de uso relevantes, STIX apunta a 
proveer una cobertura expresiva en todos sus casos de uso específicos en lugar 
de dirigirse específicamente a alguno de ellos.
  \item \underline{Integración en lugar de duplicación}: Cuando STIX abarca conceptos de información estructurada para los cuales ya 
existen representaciones estandarizadas con el consenso adecuado y que se 
encuentran disponibles, se busca integrar estas representaciones a la 
arquitectura STIX en lugar de duplicar la información innecesariamente.
\item \underline{Flexibilidad}: Con el fin de soportar un amplio rango de casos e información variable con 
varios niveles de fidelidad, se ha diseñado a STIX para ofrecer tanta flexibilidad 
como sea posible. STIX se adhiere a una política de permitir a los usuarios 
utilizar cualquier porción de representaciones estándar que sean relevantes para 
un contexto dado y evita elementos obligatorios siempre que sea posible.
\item \underline{Extensibilidad}: Con la finalidad de soportar un amplio rango de casos de uso con un potencial 
diferente de representación y para facilitar el perfeccionamiento 
impulsado por la comunidad, se ha diseñado STIX para construir mecanismos de 
extensión para usos específicos, para usos localizados, para refinamientos del 
usuario y evolución y para facilidad de refinamiento y evolución centralizada.
\item \underline{Automatización}: El diseño realizado en STIX busca maximizar la estructura y la consistencia para 
soportar métodos de procesamiento automático por máquinas.
\item \underline{Lectura}: El diseño de STIX busca estructuras del contenido para que no sea únicamente 
consumible o procesable por máquinas sino que también sea leíble por humanos. 
Esto es necesario para claridad y comprensibilidad durante las primeras etapas 
de desarrollo y adopción.
\item \underline{Implementación}: La implementación inicial de STIX utiliza XML como un mecanismo portátil, 
estructurado y que se puede encontrar en cualquier parte para la discusión, 
colaboración y refinamiento entre las comunidades involucradas. Está pensado 
para el desarrollo en colaboración de un lenguaje estructurado sobre información de 
amenazas entre expertos de la comunidad. Se ha pensado que el uso del lenguaje 
sea estimulado y soportado por medio del desarrollo de varias herramientas como 
APIs. Solo por medio de niveles adecuados de colaboración entre miembros de la 
comunidad y con la utilización de datos reales se puede llegar a que la solución 
evolucione.
\end{itemize}








