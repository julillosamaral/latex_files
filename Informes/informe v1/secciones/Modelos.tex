\subsubsection{Comunidades}
Actualmente el número de organizaciones que buscan compartir información de
amenazas es creciente. Esto lleva a que el número y tipo de comunidades que busca
hacerlo también se incremente. En la actualidad se identifican tres tipos de comunidades:
\begin{itemize}
  \item Pares
  \item Comerciales
  \item Gubernamentales
\end{itemize}

Para que se pueda compartir información entre dichas organizaciones es necesario
cierto nivel de confianza dado que compartir información sensible podría exponer 
a las organizaciones a daños en su reputación, demandas o advertir a un 
atacante de la investigación que se lleva a cabo haciendo que el trabajo 
realizado haya sido inútil. Se deben definir medidas para la protección de los 
datos como restricciones en su manejo, sanitización y el establecimiento de 
confianza entre las dos organizaciones. Lo mencionado anteriormente es 
particularmente importante cuando las organizaciones forman parte de varias 
comunidades que intercambian información, se encuentran casos en los que datos 
compartidos con una organización no deberían ser compartidos con otra.\\

Las comunidades entre \textbf{pares} son las más comunes, estas organizaciones o 
individuos tienen el propósito común de mejorar las defensas colectivas contra 
adversarios que tienen en común. La información compartida por dichas 
organizaciones es más especifica que la provista por organizaciones comerciales.\\

Las comunidades \textbf{comerciales} son anónimas y los miembros poseen algún tipo de 
acuerdo común, por ejemplo el pago de cuotas para pertenecer a la comunidad. 
La organización comercial maneja la información de forma centralizada y la 
distribuye entre los miembros de la organización. El acceso a la información por 
parte de los socios es rápido teniendo la posibilidad de que dicha información 
sea más amplia que la compartida por pares y que además no siempre sea aplicable a las 
necesidades de la organización.\\

Las comunidades \textbf{gubernamentales} son establecidas y manejadas por el gobierno, 
son voluntarias u obligatorias e incluyen participantes tanto del gobierno como 
de la industria privada. En ellas el gobierno controla la información y la 
distribución de esta. Así como en las comunidades comerciales, la información y 
los participantes son altamente confidenciales.\\

\subsubsection{Modelos}

Se pueden identificar tres modelos para el intercambio de información entre 
organizaciones:
\begin{itemize}
  \item Hub and Spoke
  \item Peer to peer
  \item Source/subscriber
\end{itemize}

En el método \textbf{hub and spoke}, la entidad hub controla la recepción y diseminación 
de los datos, además se encarga de mantener anónimos y proveer un análisis 
adicional de los datos recolectados para luego diseminarlos entre los 
participantes. Este modelo es comúnmente visto en comunidades de gobierno o comerciales.\\

En el modelo \textbf{peer to peer}, los participantes intercambian y reciben información 
directamente de los otros participantes. La información es compartida entre 
todos los miembros de la comunidad por igual y la fuente está claramente 
identificada.\\

\textbf{Source/subscriber} es utilizado por las 
comunidades comerciales que proveen de información. El proveedor de información 
envía regularmente información a todos los subscriptores y estos podrían 
eventualmente enviarle información a la fuente. Generalmente, la información se 
codifica de una manera propietaria y puede faltar información esencial sobre 
algunos intentos de irrupción. Presenta la ventaja de que se tiene acceso rápido 
a un conjunto de datos amplio y es útil para organizaciones con recursos 
limitados.\\