\subsubsection{IDMEF}

Intrusion Detection Message Exchange Format (IDMEF) fue especificado como un 
protocolo experimental que no especifica ningún estándar. El propósito de 
IDMEF es definir formatos de datos y procedimientos de intercambio para 
compartir información de interés con sistemas de detección de intrusos y 
sistemas de respuesta con los sistemas de administración que deben 
interactuar con estos. El RFC de IDMEF describe el modelo de datos para 
representar información tomada de los sistemas de intrusión. Se busca que dicho 
formato pueda ser utilizado por los Intrusion Detection Systems (IDSs) para 
reportar alertas sobre eventos que parezcan sospechosos.

Por medio de la firma digital de XML se busca proveer integridad, autenticidad de 
los mensajes y de los servicios. La responsabilidad para la integridad y 
autenticación de los mensajes es responsabilidad del protocolo de comunicación y 
no del formato de mensajes. La inclusión de firmas digitales en mensajes IDMEF 
debería ser realizada en casos en los que éstas deban ser archivadas para su uso 
posterior o cuando los mensajes IDMEF son intercambiados sobre protocolos poco 
seguros.

\paragraph{Modelo de datos de IDMEF}
El modelo de datos de IDMEF es una representación orientada a alertas enviadas a 
los administradores desde los sistemas de intrusión. El modelo de datos presenta 
varios problemas referentes a la representación de los datos:
\begin{itemize}
  \item Generalmente, la información de las alertas es heterogénea. Algunas 
  alertas son definidas con poca información, otras proveen información más 
  detallada. Ejemplos de alertas con poca información son aquellas que solo 
  presentan datos de origen, destino, nombre u hora, estos datos pueden set 
  extendidos por alertas detalladas en las cuales se presenta información de 
  usuarios, procesos o puertos de servicios. Es necesario que el modelo de datos 
  que represente dicha información sea flexible para adaptarse a las distintas 
  necesidades. Un modelo de datos orientado a objetos es extensible por medio de 
  agregación y sub clases, de esta forma una implementación que no entienda 
  estas extensiones podrá seguir entendiendo el subconjunto de información que 
  está definido en el modelo. Estas dos formas de extender el modelo permiten 
  que se mantenga su consistencia.
  \item Los IDSs son diferentes, algunos de estos detectan ataques analizando el 
  tráfico en la red, otros utilizando los logs de sistemas operativos o 
  aplicaciones que auditan información. Las alertas generadas para un mismo 
  ataque pero por medio de diferentes herramientas, no necesariamente contienen 
  los mismos datos. El modelo de datos define clases que soportan las 
  diferencias en las fuentes de los datos. En particular, las nociones de 
  fuente y objetivo para la alerta son representadas por una combinación de 
  nodo, procesos, servicio y clases de usuario.
  \item Dependiendo del tipo de red o sistema operativo utilizado, los 
  ataques serán observados y reportados de manera diferente lo cual lleva a que 
  el modelo de datos deba adaptarse a dichas diferencias.
  \item Los instrumentos comerciales persiguen objetivos diferentes. Por varias 
  razones, estos desean entregar más o menos información sobre ciertos tipos de 
  ataques. El modelo debe permitir la flexibilidad necesaria preservando su 
  integridad.
\end{itemize}

El diseño del modelo de datos busca proveer una representación estándar de las 
alertas de manera que la información no sea ambigua y que se describa la 
relación entre alertas simples y complejas.

El objetivo de dicho modelo es proveer una representación estándar de la 
información analizada por un sistema de intrusión cuando es detectado 
un evento inusual. Estas alertas pueden ser simples o complejas 
dependiendo de las capacidades de la herramienta que las creo.

Los nuevos objetos son introducidos para referenciar contenido adicional. Esto 
es importante debido a que la tarea de clasificar y nombrar vulnerabilidades es 
difícil y sumamente subjetiva. El modelo de datos no debe ser ambiguo, por lo 
que mientras que se permite que los análisis sean más o menos precisos 
entre ellos, no se debe permitir que estos produzcan información contradictoria  
en dos alertas que describen el mismo evento. De todas formas, siempre es 
posible insertar toda la información útil de un evento en campos de extensión 
de la alerta en lugar de en los campos a los que estos pertenecen, sin embargo, 
dichas prácticas reducen la interoperabilidad y deberían ser evitadas en lo 
posible.


