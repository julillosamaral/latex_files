\subsubsection{RID}

\textit{Real-Time Inter-Network} (RID) es un método de comunicación entre redes para 
facilitar el intercambio de datos de incidentes. A su vez, busca integrar 
mecanismos existentes de detección, seguimiento, identificación de fuentes y 
mitigación que aporta una solución al manejo de incidentes. Combinar estas 
capacidades en un sistema de comunicación permite incrementar el nivel de 
seguridad en la red. Las políticas para manejar incidentes son recomendadas y 
pueden ser acordadas por un consorcio utilizando recomendaciones y 
consideraciones de seguridad.\\

RID ha sido ampliamente utilizando en comunidades de investigación, pero no ha 
sido muy adoptado por otros sectores. Fue desarrollado como un mecanismo de 
comunicación para facilitar la transferencia de información entre distintos 
proveedores de servicios de Internet para trazar precisa y eficientemente el 
flujo de paquetes nocivos a lo largo de la red. RID considera la información que 
necesitan varias implementaciones para seguir paquetes dentro de una red y los 
requerimientos de los proveedores de servicios de decidir si se permite trazar 
el recorrido de un paquete o no.\\

Los datos en RID son representados como documentos XML utilizando IODEF. De esta 
forma se simplifica la integración con otros aspectos del manejo de incidentes. 
Además las ventajas de la representación en XML permite conservar la privacidad 
y tener un buen nivel de aseguramiento. RID busca proveer un método para 
comunicar la información relevante entre CSIRTs manteniendo la compatibilidad 
con varios sistemas de rastreo y respuesta.\\

Los mensajes RID son encapsulados en otros protocolos para el transporte, este 
procedimiento se define en el RFC 6046. La autenticación, integridad y 
autorización son el resultado de las capacidades de cada capa y son utilizadas 
para alcanzar un nivel de aseguramiento adecuado.\\

Los mensajes RID tienen la intención de ser usados en el manejo coordinado de 
incidentes para localizar la fuente de un ataque y detener o mitigar sus 
efectos. Los objetivos de los ataques incluyen redes o sistemas que se vieron 
comprometidos con ataques como denegaciones de servicio u otro tipo de tráfico 
malicioso en una red. Por medio de RID los CSIRTs tienen la posibilidad de 
reportar ataques que estén ocurriendo a otros CSIRTs o pedir información a éstos 
de la detección de ataques.\\

Los sistemas comprometidos pueden ser el resultado de otros incidentes de 
seguridad como worms, troyanos o virus. El manejar incidentes es una tarea 
difícil para algunas organizaciones debido al tamaño de su red y la cantidad de 
recursos disponibles. RID provee el framework necesario para realizar la 
comunicación entre redes involucradas en el manejo, seguimiento y mitigación de 
incidentes de seguridad. Distintos tipos de mensajes son necesarios para 
facilitar el manejo de incidentes. Los mensajes que se incluyen son 
\textit{Report}, \textit{IncidentQuery}, \textit{TraceRequest}, 
\textit{RequestAuthorization}, \textit{Result} e \textit{InvestigationRequest}. El 
mensaje Report es utilizado cuando se ingresa un incidente en un sistema RID y 
no se deben realizar más acciones. Un mensaje Incident Request es utilizado para 
pedir información de un incidente en particular. Un mensaje Trace Request es 
utilizando cuando la fuente del tráfico puede estar oculta. En ese caso, cada 
proveedor de red que reciba uno de estos mensajes enviará uno a la red anterior 
en el camino del mensaje para obtener la fuente del tráfico. Los mensajes 
Request Authorization y Result son utilizados para comunicar el estado y 
resultado de un mensaje Trace Request o Investigation Request. El mensaje 
Investigation Request solo considera lo sistemas RID en el camino a la fuente 
del tráfico.\\

En RID se pueden diseñar topologías que permitan el intercambio de información 
facilitando la comunicación entre socios. La topología básica para comunicar 
sistemas RID es la de una comunicación directa. En una organización se pueden 
establecer y utilizar varias topologías. Una podría fortalecer las relaciones 
bilaterales entre socios. Los socios podrían reenviar los mensajes RID entre 
ellos. Este enfoque permite un rastreo iterativo en donde la fuente es 
desconocida.\\

La comunicación entre los sistemas RID debe ser protegida. RID tiene muchas 
consideraciones de seguridad incluidas en el diseño del protocolo. Al considerar 
el transporte de mensajes RID, una red fuera de banda, ya sea física o lógica, 
puede prevenir ataques externos contra las comunicaciones RID. Se deben utilizar 
conexiones autenticadas y encriptadas entre sistemas RID para proveer 
confidencialidad, integridad, autenticidad y privacidad de los datos. Las 
relaciones de confianza son realizadas por socios y establecen relaciones de 
confianza por medio de infraestructura de PKI.\\

El protocolo de transporte utilizado debe proveer encriptación para dar un nivel 
de seguridad e integridad adicional, mientras se provee autenticación por medio 
de certificados. De todas formas los mensajes RID no utilizan únicamente la 
seguridad provista por el protocolo de transporte.\\

La infraestructura PKI provee la base para la autenticación, autorización, 
encriptación y firmas digitales necesarias para establecer una relación de 
confianza entre los miembros de una comunidad RID.\\

Problemas de privacidad pueden ser de preocupación cuando se habla de compartir 
información y deben ser considerados a la hora de alcanzar la meta de detener o 
mitigar los efectos de un incidente de seguridad. Para ello hay clases 
específicas para automatizar las políticas de privacidad.\\

IODEF define un formato de mensaje, no un protocolo de transporte, esto se 
realiza para permitir a los CSIRTs intercambiar y almacenar los datos de la 
forma que más le conviene a cada organización. Sin embargo, RID requiere una 
especificación de un protocolo de transporte para asegurar la interoperabilidad 
entre las organizaciones socias. El RFC6046 mencionado anteriormente, especifica 
el transporte de mensajes RID sobre HTTPS/TLS. En esta especificación, cada 
servidor RID funciona como servidor y cliente. Todos los sistemas RID deben 
estar preparados para aceptar conexiones HTTP/TLS de cualquier socio con la 
finalidad de soportar llamadas a respuestas tardías de pedidos que se 
realizaron.












