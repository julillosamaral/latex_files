\makeglossary
\glossary{name={Botnet}, description={Es una colección de computadoras comprometidas (llamadas computadoras zombie), 
		instaladas generalmente mediante gusanos, troyanos y backdoors, controladas 
		remotamente, usualmente con fines maliciosos, en especial para denegaciones de 
		servicio.}}

%cambiar a usar glosario posta
\glossary{name={CAPEC}, description={Es una lista pública desarrollada por la comunidad de patrones de 
		ataque comunes así como un esquema comprensivo junto con una clasificación.}}
\glossary{name={CSIRT}, description={Es una organización responsable de recibir reportes de incidentes de seguridad, analizarlos y responder a ellos. [CERT/CC]}}
\glossary{name={CVE}, description={El CVE (\textit{Common Vulnerabilities and Exposures}) es un diccionario de comunes (por 
		ejemplo los identificadores CVE) para publicar información conocida de 
		vulnerabilidades de seguridad. El CCE (\textit{Common Configuration Enumeration}) provee 
		identificadores para problemas de configuración de seguridad. Los 
		identificadores hacen más fácil compartir información a través de diferentes 
		bases de datos y herramientas de seguridad.}}
\glossary{name={Cyber Observables}, description={}}

\glossary{name={Cyber Threat information}, description={Es cualquier información representable en STIX. 
		Esto incluye, pero no esta limitado a, observables, indicadores, incidentes, 
		TTPs (\textit{Tactics}, \textit{Techniques} y \textit{Procedures}), \textit{exploit targets}, \textit{campaigns}, \textit{threat 
			actors} y \textit{courses of action}.}}
\glossary{name={TAXII Data Feed}, description={Es una colección de \textit{cyber threat information} 
		estructurada expresable en uno o mas documentos STIX que pueden ser 
		intercambiados utilizando TAXII. Cada TAXII Data Feed debe tener un 
		nombre que lo identifica de forma única entre el resto de los feeds de un 
		productor dado. Cada elemento de un TAXII data feed debe ser etiquetado con un 
		timestamp y puede tener otras etiquetas a discreción del productor.}}
 \glossary{name={IDS (Intrusion Detection System)}, description={Sistema que detecta intrusiones no deseadas en una 
 		red o equipo, que no pueden ser detectadas por un \textit{firewall} convencional.}}
 
\glossary{name={MAEC}, description={Es un lenguaje estandarizado para codificar y comunicar 
		información de \textit{malware} basandose en atributos como comportamiento, patrones de 
		ataque, etc.}}
  \glossary{name={Mensaje TAXI}, description={Un bloque de información que es pasado de una entidad a la 
  		otra. Un mensaje TAXII representa un pedido o una respuesta.}}
 
 \glossary{name={Intercambio de mensajes TAXI}, description={Una secuencia definida de mensajes TAXII 
 		intercambiados entre dos entidades.}}
 \glossary{name={Servicio TAXII}, description={Son funcionalidades albergadas por algunas entidades y que 
 		es accedido o invocado usando uno o mas TAXII Message Exchange.}}

\glossary{name={TAXII Capabilit}, description={Una actividad de alto nivel soportado por TAXII por 
		medio del uso de uno o mas servicios TAXII.}}

%\newpage
