\subsection{Interacción con la comunidad de Diaspora}

En esta sección se explicarán la interacción que hubo a lo largo del desarrollo de este proyecto con la comunidad de \emph{Diaspora} y las herramientas utilizadas para la misma.

\subsubsection{Github}

Durante el desarrollo del proyecto hubo varios intercambios con la comunidad de \emph{Diaspora} y especialmente con los responsables del repositorio. En particular todas las 
contribuciones de código que se realizan a Diaspora se deben realizar a través de pull requests al repositorio oficial, el cual se encuentra hosteado en GitHub \cite{github}. Las 
mismas pueden ser revisadas y comentadas por cualquier desarrollador, pero sólo los miembros del \emph{Core Team} pueden aprobarlas e incluirlas a la rama principal de 
desarrollo.

Esta modalidad de trabajo permite que las contribuciones sean de la mayor calidad posible ya que todo tipo de desarrolladores pueden realizar revisiones de código y sugerir 
cambios. Uno de los principales problemas que sufrió \emph{Diaspora}, fue que una vez que los creadores del proyecto abandonaron el desarrollo del mismo, muchas de las 
contribuciones se aceptaban por más que las mismas no fueran de buena calidad y no tuvieran buen cubrimiento de pruebas. La ideología detrás de esto era que todos los aportes eran 
bienvenidos y que después alguno de los responsables del repositorio se iba a encargar de arreglar el código introducido. Ésta es la principal razón por la cual el código de 
\emph{Diaspora} se encuentra en el estado actual. Desde hace un tiempo esta política se revirtió y todos las contribuciones son revisadas y aceptadas por algún miembro del 
\emph{Core Team} solamente si tienen pruebas asociadas y el código es de calidad.

A lo largo del desarrollo del proyecto se crearon dieciséis pull requests, que proveen mejoras al proyecto y las mismas fueron aceptadas por los responsables del mismo. Algunas de 
estas mejoras ya se encuentran actualmente en producción y otras formarán parte del próximo release de la aplicación.

\subsubsection{IRC}

\emph{Diaspora} cuenta con cinco canales de IRC \cite{irc}. El canal principal \#diaspora es utilizado para realizar discusiones generales sobre \emph{Diaspora} y para ayudar a la 
gente a instalar la aplicación. Muchas de las discusiones sobre funcionalidades nuevas son discutidas en este canal. En cambio, \#diaspora-dev es utilizado para discutir sobre el 
código fuente y para ayudar a desarrolladores a realizar contribuciones. Se invirtió mucho tiempo en este canal, debido a las contribuciones que se realizaron, muchas de las 
discusiones que tuvieron lugar facilitaron la aceptación de las contribuciones que se hicieron, ya que el código de las mismas seguía los estándares establecidos por \emph{Diaspora}. 
Los otros tres canales son para discusiones en diversos idiomas, español, alemán y francés.

\subsubsection{Loomio}

Loomio  es una herramienta open-source para tomar decisiones de forma colectiva. La misma permite realizar discusiones y votaciones entre otras funcionalidades. \emph{Diaspora} 
cuenta con tres grupos en esta aplicación: \emph{Feature Proposals}, \emph{Developers Proposals} y \emph{Governance Proposals}. En ellos se crean discusiones y votaciones 
respecto a agregar nuevas funcionalidades a la aplicación, aspectos técnicos de la misma y aspectos administrativos de \emph{Diaspora} respectivamente. La participación en los 
grupos de \emph{Diaspora} por el momento no son abiertos y se debe solicitar un invitación para los mismos, para ello es necesario enviar un mail a uno de los responsables del repositorio.

Debido a que el proyecto quedó en manos de la comunidad, todas las decisiones importantes son tomadas mediante el uso de esta aplicación. Las discusiones involucran distintos temas, desde  restructuras importantes de código, cambios en las políticas de privacidad, e incluso hasta la definición de flujos de trabajo.

Una de las últimas votaciones realizadas fue la de la inclusión de uno de los integrantes del grupo al \emph{Core Team} de \emph{Diaspora}. La misma se encuentra en 
\cite{loomio_fabian}. En la misma participaron los integrantes del \emph{Core Team}, y votaron la inclusión de \emph{fabianrbz} al mismo. Esta votación fue propuesta por uno de los integrantes más activos del \emph{Core Team} y se debió a la cantidad de contribuciones que realizó y a la calidad de las mismas.

