% Chapter Template

\chapter{STIX} 
\label{Chapter3}
\lheader{Chapter 3. \emph{STIX}} % Change X to a consecutive number; this is for the header on each page - perhaps a shortened title

%----------------------------------------------------------------------------------------
%	SECTION 1
%----------------------------------------------------------------------------------------

\section{Background}

STIX busca dar una representación estructurada de la información y que a su vez 
está sea expresiva, flexible, extensible, automatizable y legible. Esto se 
presenta como un desafío ya que la información intercambiada debe ser 
estructurada para que pueda ser parseada por una maquina pero no se debe perder 
la posibilidad de que una persona la pueda leer y entender así de esta forma no se 
pierde el juicio y control que esta puede tener.

También se busca que STIX se transforme en un estándar para la comunidad para 
obtener información de calidad que de un aprovechamiento mejor de esta y sin 
saber de antemano la fuente de los datos provistos.

El desarrollo de una herramienta de este tipo se ha vuelto urgente dado la 
velocidad y complejidad con la cual evolucionan las amenazas también teniendo 
consideración de la basta cantidad de datos que se deben intercambiar. Por ello 
es necesario establecer una forma automática para ayudar a los analistas a 
ejecutar acciones defensivas con un conjunto de datos mas grande del que podrían 
obtener si trabajaran aisladamente.

\section{Aproximaciones de la actualidad}
La información que se intercambia y utiliza hoy en día es atómica, 
inconsistente y muy limitada en sofisticación y expresividad. En donde se 
utilizan estructuras estandarizadas, estas son tipicamente focalizadas en una 
porción del problema. Además no se integran de forma adecuada entre si o carecen 
de flexibilidad. Las actividades de intercambio de indicadores son entre 
humanos, siendo los indicadores desestructurados o semi-estructurados 
intercambiados por portales web o encriptados vía email. Más recientemente se 
ha visto el surgimiento de transferencias entre maquinas de conjuntos de 
indicadores simples de modelos de ataques bien conocidos.

A diferencia de IODEF, STIX posee una lista de elementos para construir 
indicadores de compromiso y se integra con CAPEC, MAEC o CVE. Además posee 
soporte para tácticas, técnicas y procedimientos del adversario y campañas 
realizadas por la organización. IODEF fue pensado para compartir información de 
incidentes y no indicadores de compromiso. STIX permite el intercambio de 
indicadores referentes a amenazas así como información de contexto. Juntos, 
permiten tener un conjunto más completo de las intenciones, capacidades, 
motivaciones y actividades de un adversario y de esa forma defenderse de él.

STIX busca extender los indicadores para permitir el manejo e intercambio de 
indicadores de forma más expresiva así como de un espectro más amplio de 
información.

Actualmente, el intercambio y manejo de información de forma automática de 
información es visto típicamente en lineas de productos, servicios ofrecidos o 
soluciones especificas de una comunidad. STIX busca permitir el intercambio de 
información de forma comprensiva, rica, de alta fidelidad entre organizaciones, 
comunidades, productos y servicios ofrecidos. STIX es un lenguaje para la 
especificación, captura, caracterización y comunicación de información de 
seguridad estándar. Lo hace de forma estructurada para soportar un mejor manejo 
de la información y la aplicación de automatización. Una variedad de casos de 
uso para dicha información son realizados.
\begin{itemize}
  \item Análisis de las amenazas
  \item Especificación de patrones de indicadores para la amenaza.
  \item Manejo de las actividades de respuesta
  \item Intercambiar información
\end{itemize}

STIX provee un mecanismo común para hacer frente a estos casos de uso dando 
consistencia, eficiencia, interoperabildiad y conciencia global de la situación.

STIX provee una arquitectura unificada juntando un conjunto amplio de 
información incluyendo:
\begin{itemize}
  \item Cyber Observables
  \item Indicadores
  \item Incidentes
  \item Tácticas, técnicas y procedimientos de los adversarios
  \item Objetivos de exploits
  \item Cursos de acción
  \item Campañas de ataques
  \item Actores de ataques
\end{itemize}

Para permitir que dicha solución sea practica para cualquier caso de uso, 
lenguajes existentes y estandarizados son utilizados.

\section{Casos de uso}
\subsection{Análisis de amenazas}
Un analista revisa información estructurada y no estructurada respecto a 
actividades de amenazas de una variedad de fuentes de entradas manuales o 
automáticas. El analista busca entender la naturaleza de las amenazas 
relevantes, identificarlas y caracterizarlas totalmente para que todo el 
conocimiento relevante de la amenaza sea totalmente expresado y evolucionado a 
traves del tiempo. Este conocimiento relevante incluye acciones relacionadas con 
la amenaza, comportamientos, capacidades, intenciones, actores atribuidos, etc. 
Por medio del entendimiento y la caracterización, el analista puede especificar 
patrones de indicadores relevantes, sugerir acciones para las actividades de 
respuesta y compartir la información con miembros de la comunidad.

\subsection{Especificación de patrones de indicadores}
Un analista especifica patrones medibles representando las características 
observables de una amenaza junto con su contexto y metadata para ser 
interpretadas, manejadas y aplicar el patrón y sus resultados. Esto puede ser 
realizado de forma manual o con la asistencia de una herramienta automatizada.

\subsection{Manejo de las actividades de respuesta}
Los tomadores de decisión y el personal de operaciones trabajan en conjunto para 
prevenir o detectar actividades amenazantes y responder a los incidentes 
detectados que sean realizados por dichas amenazas. Las acciones preventivas 
pueden mitigar vulnerabilidades, debilidades, o malas configuraciones que sean 
que sean objetivos de los exploits. Luego de la detección e investigación de 
incidentes específicos, acciones reactivas pueden ser realizadas.

\subsubsection{Prevención de amenazas}

Los tomadores de decisiones evalúan acciones preventivas para amenazas 
relevantes que sean identificadas y seleccionan acciones apropiadas para su 
implementación. El personal de operaciones implementa las acciones seleccionadas 
para prevenir la ocurrencia de amenazas especificas aplicando mitigaciones 
amplias o con objetivos específicos iniciados por interpretación de los 
indicadores.

\subsubsection{Detección de amenazas}

El personal de operaciones aplica mecanismos (automáticos y manuales) para 
monitorear y asistir las operaciones con la finalidad de detectar la ocurrencia 
de amenazas específicas basándose en la evidencia histórica, análisis del 
contexto actual, interpretación de los indicadores que se presentan. Esta 
detección se realiza generalmente por medio de patrones de indicadores.

\subsubsection{Respuesta a incidentes}

El personal de operaciones responde a amenazas detectadas, investiga que ha 
ocurrido o está ocurriendo, trata de identificar y caracterizar la naturaleza de 
las amenazas y lleva a cabo la mitigación o lleva a cabo cursos de acción 
preventivos. Una vez que los efectos son entendidos, el personal de operaciones 
puede implementar mitigaciones acordes o llevar a cabo cursos de acción 
correctivos.

\subsection{Compartir información de amenazas}
Los tomadores de decisión establecen políticas respecto a que tipo de 
información será compartida, además toman decisiones respecto a con quienes se 
comparte y como debería ser manejada basándose en frameworks de confianza de 
forma de mantener niveles adecuados de consistencia, contexto y control. Esta 
política es luego implementada para compartir los indicadores de amenazas 
apropiados y otra información de amenazas.

\section{Principios de guía}
En el enfoque dado a STIX se ha buscado implementar un conjunto de principios 
con el consenso de la comunidad. Esos principios son los siguientes:
\subsection{Expresividad}
Con el fin de soportar la diversidad de casos de uso relevantes, STIX apunta a 
proveer una cobertura expresiva en todos sus casos de uso específicos en lugar 
de dirigirse específicamente a alguno de ellos.
\subsection{Integración en lugar de duplicación}
Cuando STIX abarca conceptos de información estructurada para los cuales ya 
existen representaciones estandarizadas con el consenso adecuado y que se 
encuentran disponibles, se busca integrar estas representaciones a la 
arquitectura STIX en lugar de duplicar esta información innecesariamente.
\subsection{Flexibilidad}
Con el fin de soportar un amplio rango de casos e información variable con 
varios niveles de fidelidad, STIX esta diseñado para ofrecer tanta flexibilidad 
como sea posible. STIX se adhiere a una política de permitir a los usuarios 
utilizar cualquier porción de representaciones estándar que sean relevantes para 
un contexto dado y evita elementos obligatorios siempre que sea posible.

\subsection{Extensibilidad}
Con la finalidad de soportar un amplio rango de casos de uso con un potencial 
diferente de representación y para facilitar el perfeccionamiento 
impulsado por la comunidad, se ha diseñado STIX para construir mecanismos de 
extensión para usos específicos, para usos localizados, para refinamientos del 
usuario y evolución y para facilidad de refinamiento y evolución centralizada.

\subsection{Automatización}
El diseño realizado en STIX busca maximizar la estructura y la consistencia para 
soportar métodos de procesamiento automático por máquinas.

\subsection{Lectura}
El diseño de STIX busca estructuras del contenido para que no sea únicamente 
consumible o procesable por maquinas sino que también sea leíble por humanos. 
Esto es necesario para claridad y comprensibilidad durante las primeras etapas 
de desarrollo y adopción.

\section{Implementación}
La implementación inicial de STIX utiliza XML como un mecanismo portable, 
estructurado y que se puede encontrar en cualquier parte para la discusión, 
colaboración y refinamiento entre las comunidades involucradas. Esta pensado 
para el desarrollo colaborativo de un lenguaje estructurado sobre información de 
amenazas entre expertos de la comunidad. Se ha pensado que el uso del lenguaje 
es incentivado y soportado por medio del desarrollo de varias herramientas como 
APIs. Solo por medio de niveles adecuados de colaboración entre miembros de la 
comunidad y con la utilización de datos reales se puede llegar a que la solución 
evolucione.



