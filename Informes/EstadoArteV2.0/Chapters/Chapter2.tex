% Chapter Template

\chapter{Métodos actuales} % Main chapter title

\lheader{IDMEF e IODEF} % Change X to a consecutive number; this is for the header on each page - perhaps a shortened title

\lfoot{Chapter 1. \emph{Protocolos actuales}}
%----------------------------------------------------------------------------------------
%	SECTION 1
%----------------------------------------------------------------------------------------

\section{IDMEF}
IDMEF (Intrusion Detection Message Exchange Format) fue especificado como un protocolo experimental que no especifica ningún 
estándar. Sin embargo IODEF se basa en las especificaciones de las clases 
definidas en IDMEF. El propósito de IDMEF es definir formatos de datos y 
procedimientos de intercambio para compartir información de interés con sistemas 
de detección de intrusos y sistemas de respuesta con los sistemas de 
administración que podrían necesitar interactuar con estos. El RFC de IDMEF 
describe el modelo de datos para representar información tomada de los sistemas 
de intrusión y explica la razón de usar este modelo. IDMEF busca ser un formato 
que pueda ser utilizado por los IDSs para reportar alertas sobre eventos que 
parecen sospechosos. Se puede proveer interoperabilidad entre sistemas 
comerciales, open source y sistemas de investigación.%, permitir a los usuarios 
%esta mezcla de sistemas ayuda a obtener una implementación óptima de sus sistemas.


Se utiliza XML digital signature para proveer integridad, 
autenticación de los mensajes y  de los servicios del firmante para 
datos de cualquier tipo, sean estos localizados en el XML que incluye la firma o 
en otro lugar. La responsabilidad para la integridad y autenticación de  los 
mensajes es responsabilidad del protocolo de comunicación y no del formato de 
mensaje. Sin embargo en situaciones en que los mensajes IDMEF son intercambiados 
sobre protocolos poco seguros, o en casos en que las firmas digitales deban ser 
archivadas para uso posterior, la inclusión de firmas digitales en mensajes 
IDMEF debería ser realizada.

\subsection{Modelo de datos de IDMEF}
El modelo de datos de IDMEF es una representación orientada a objetos de las 
alertas enviados a los administradores desde los sistemas de intrusión.
El modelo de datos presenta varios problemas referentes a la representación de 
los datos:
\begin{itemize}
  \item La información de las alertas es mayormente heterogénea. Algunas alertas son 
  definidas con muy poca información, como orígenes, destinos, nombre u hora del 
  evento. Otras alertas proveen mucha mas información, como puertos de 
  servicios, procesos, información de usuarios, etc. El modelo de datos que debe 
  representar dicha información debe ser flexible para adaptarse a las distintas 
  necesidades. Un modelo orientado a objetos es extensible por medio de 
  agregación y sub clases. Si una implementación del modelo de datos extiende 
  con una nueva clase, por medio de agregación o subclases, una implementación 
  que no entienda estas extensiones podrá seguir entendiendo el subconjunto de 
  información que esta definido en el modelo. Estas dos formas de extender el 
  modelo permiten que se mantenga la consistencia del modelo.
  \item Los IDS son diferentes, algunos analizadores detectan ataques analizando 
  el trafico en la red, otros utilizando los logs de sistemas operativos o 
  aplicaciones que auditan información. Alertas para el mismo ataque enviadas 
  por analizadores con diferentes fuentes de información, no contendran los 
  mismos datos.
  El modelo de datos define clases que soportan las diferencias en las fuentes 
  de los datos. En particular, las nociones de fuente y objetivo para el alerta 
  son representadas por una combinación de nodo, procesos, servicio y clase de 
  usuario.
  \item Los ambientes operacionales son diferentes. Dependiendo de el tipo de 
  red o sistemas operativos utilizados, los ataques serán observados y 
  reportados con diferentes características. El modelo de datos se 
  adapta a estas diferencias.
  \item Los instrumentos comerciales persiguen objetivos diferentes. Por varias 
  razones, estos desean entregar mas o menos información sobre ciertos tipos de 
  ataques. El modelo orientado a objetos permite la flexibilidad necesaria 
  preservando la integridad del modelo.
\end{itemize}

\subsection{El diseño del modelo de datos}

El modelo de datos fue diseñado para proveer una representación estándar de las 
alertas de una manera que la información no sea ambigua. Además busca permitir 
describir la relación entre alertas simples y complejas.

\subsubsection{Representación de eventos}
El objetivo del modelo de datos es proveer una representación estándar de la 
información que el analizador de un sistema de intrusión reporta cuando detecta 
la ocurrencia de eventos inusuales. Estas alertas pueden ser simples o complejas 
dependiendo de las capacidades del analizador que la creo.

\subsubsection{Enfocada al contenido}

El diseño del modelo de datos es enfocado al contenido. Lo cual significa que 
los nuevos objetos son introducidos para referenciar a contenido adicional. Esto 
es importante debido a que la tarea de clasificar y nombrar vulnerabilidades es difícil y sumamente subjetiva. El modelo de datos no debe ser 
ambiguo, esto significa que mientras se permite que los analizadores sean mas o 
menos precisos entre ellos, no se debe permitir que estos produzcan información 
contradictoria en dos alertas describiendo el mismo evento. De todas formas, 
siempre es posible insertar toda la información ''interesante'' de un evento en 
campos de extensión de la alerta en lugar de a los campos a los que estos 
pertenecen, sin embargo, dichas practicas reducen la interoperabilidad y 
deberian ser evitadas en lo posible.

\subsubsection{Relación entre alertas}

IDMEF busca cubrir todo el rango de alertas posibles, desde las simples a las 
más complejas. Para esto el modelo de datos debe proveer una forma de 
representar alertas complejas que agregan muchas alertas simples y que se 
permitan identificar dichas alertas simples en el contenido de la alerta 
compleja.

\section{IODEF}

Las organizaciones deben colaborar entre ellas para mitigar las actividades 
maliciosas que atacan sus redes así como para ganar conocimiento de posibles 
amenazas. Esta coordinación puede requerir coordinación con ISPs, sitios remotos 
o intercambiar datos con socios.

IODEF son las siglas para Incident Object Description Exchange Format, este 
define una representación de datos que provee un framework para el intercambio 
de información entre CSIRTs, dicho tipo de información es intercambiado 
comúnmente por CSIRTs y es referente a incidentes de seguridad. IODEF provee una 
representación en XML para transportar información de incidentes entre dominios 
administrativos entre pares que tienen una responsabilidad operacional de 
remediar o de analizar y establecer advertencias en un dominio definido. El 
modelo de datos provisto en IODEF codifica la información referente a hosts, 
redes y servicios corriendo en estos sistemas; metedologia de ataques y 
evidencia forense asociada; impacto de la actividad realizada; aproximaciones 
limitadas para documentar el flujo.

IODEF debería ser compatible con IDMEF y ser capaz de user e incluir mensajes 
IDMEF. Los actores principales de en IODEF son lo CSIRTs y no los IDs como 
sucede en IDMEF. Se puede ver IODEF como una interfaz orientada ser utilizada 
por personas, para ello un mensaje IODEF es parseable por una maquina pero 
leíble por humanos. Los objetos de IODEF tienen un tiempo de vida mayor a IDMEF, 
en este los mensajes son utilizados una sola vez. Los mensajes IODEF consideran 
información referente a manejo de incidentes, el almacenamiento de dicha de 
información o estadísticas y análisis.

El objetivo primordial de IODEF es mejorar las capacidades operacionales de los 
CSIRTs. La adopción por parte de la comunidad provee una habilidad mejorada para 
resolver los incidentes y transmitir el contexto  simplificando la colaboración 
e intercambio de datos. El formato estructurado provisto por IODEF permite:
\begin{itemize}
  \item Incrementar la automatización en el procesamiento de los datos, ya que 
  los recursos de los analistas de seguridad de parsear documentos se verá 
  reducido.
  \item Bajar el esfuerzo necesario para normalizar datos similares de 
  diferentes fuentes.
  \item Un formato común con el cual construir herramientas interoperables para 
  el manejo de incidentes y análisis subsecuente, específicamente cuando los 
  datos provienen de dominios distintos.
\end{itemize}

La coordinación entre CSIRTs no es un problema estrictamente técnico. Hay muchas 
consideraciones: procedimientos, confianza, legales, las cuales pueden ocasionar 
que las organizaciones intercambien información. IODEF no busca evadir dichas 
consideraciones, sin embargo, las implementaciones operacionales de IODEF deben 
considerar este contexto.

\subsection{Modelo de datos de IODEF}
A la hora de diseñar IODEF se realizaron ciertas consideraciones de diseño
\begin{itemize}
  \item El modelo de datos sirve como formato de transporte. Por ello, su 
  representación especifica no es optima para almacenamiento en disco, 
  archivamiento a largo plazo o procesamiento en memoria.
  \item Por medio de la implementación no se busca establecer un consenso 
  respecto a la definición de un incidente. Se busca en su lugar un 
  entendimiento amplio que sea lo suficientemente flexible como para abarcar la 
  mayoría de las operaciones.
  \item Describir un incidente para todas las definiciones requeriría un modelo 
  de datos extremadamente complejo. Por ello, IODEF solo busca dar un marco para 
  transmitir información de incidentes comúnmente intercambiada. Se asegura un 
  mecanismo amplio para ser extendido para soportar información propia de la 
  organización, y técnicas para referenciar información mantenida por fuera del 
  modelo de datos.
  \item El dominio del análisis de seguridad no esta totalmente estandarizado y 
  debe basarse en descripciones textuales libres. IODEF busca conseguir un 
  balance entre el contenido libre y el procesamiento automático 
  de la información de los incidentes.
  \item IODEF es solo una de las representaciones que han sido estandarizadas. 
  El modelo de datos de IDMEF influencio el diseño de IODEF.
\end{itemize}

\subsection{Implementaciones de IODEF}
Las implementaciones del protocolo se especifican como un esquema XML. 
Implementar IODEF en XML provee varias ventajas. Que sea extensible lo hace 
ideal para especificar codificación de datos que soporta varias codificaciones 
de caracteres. Es mas simple la manipulación debido a la presencia de varias 
tecnologías para ello. Aunque fundamentalmente XML es una representación de 
texto, lo cual lo hace ineficiente cuando se deben embeber datos binarios o 
grandes volúmenes de datos deben ser intercambiados.

\subsection{Internacionalización}
Internacionalización y localización son de interés para IODEF, dado que solo por 
medio de colaboración (a menudo con barreras idiomáticas) son resueltos ciertos 
incidentes. IODEF soporta esto dependiendo de las construcciones de XML, y por 
medio de diseño explícito en el modelo de datos.

Como IODEF es implementado como un esquema XML, este soporta las diferentes 
codificaciones de caracteres. Además, los documentos IODEF deben especificar el 
lenguaje en el cual sus contenidos son codificados. 

\subsection{Consideraciones de Seguridad}
El modelo de datos de IODEF no introduce problemas de seguridad. Este solo 
define una representación simple para información de incidentes. Como los datos 
codificados por IODEF pueden ser considerados sensibles por las partes que los 
intercambian o por los descritos por los datos, se deben tomar precauciones para 
asegurar la confidencialidad durante el intercambio y el subsecuente 
procesamiento. El primero debe ser resguardado por un formato de mensaje, pero 
luego se deben tener consideraciones de seguridad por el sistema que procesa los 
datos, los almacena y archiva la documentación y la información derivada de 
estos.

El contenido de un documento IODEF puede incluir un pedido de acción o un parser 
puede independientemente tener lógica para realizar cierta acción basandose en 
información que encuentra. Por esta razón, se debe tener cuidado de autenticar 
apropiadamente el beneficiario del documento y atribuir un nivel de 
confidencialidad apropiado a los datos antes de realizar la acción.
El formato de mensaje subyacente  y el protocolo utilizado para intercambiar 
datos provee una garantía de confidencialidad, integridad  y autenticidad. El 
uso de protocolos de seguridad estandarizados es recomendado. Por ejemplo el uso 
de IODEF/RID.

Con el fin de sugerir buenas practicas en el  procesamiento y manejo de los 
datos codificados, IODEF  permite a un emisor de documentos transmitir una 
política de privacidad utilizando un atributo de restricción. Las distintas 
instancias de este atributo permite a los distintos elementos del documento 
tener las mismas políticas. Esto sirve como una guía para el receptor, pero este 
podría decidir ignorarlo, este problema no es un reto técnico.

\section{Protocolo RID}

Real-time Inter-Network Defense (RID) es un método de comunicación entre redes 
para facilitar el intercambio de datos de incidentes y a su vez integrando 
mecanismos existentes de detección, seguimiento, identificación de fuentes y 
mitigación que aporta una solución al manejo de incidentes. Combinar estas 
capacidad en un sistema de comunicación provee una forma de alcanzar un nivel de 
seguridad en la red. Las políticas para manejar incidentes son recomendadas y 
pueden ser acordadas por un consorcio utilizando recomendaciones y 
consideraciones de seguridad.

RID ha sido ampliamente utilizado en comunidades de investigación, pero no ha 
sido muy adoptado en otros sectores.

RID fue desarrollado como un mecanismo de comunicación para facilitar la 
transferencia de información entre distintos proveedores de servicios de 
internet para trazar precisa y eficientemente el flujo que llevan paquetes 
nocivos a lo largo de la red. RID considera la información que necesitan 
varias implementaciones para seguir paquetes dentro de una red y los 
requerimientos de los proveedores de servicios de decidir si se permite trazar el 
recorrido de un paquete o no.

Los datos en RID son representados como documentos XML utilizando el formato de 
IODEF. De esta forma se simplifica la integración con otros aspectos del manejo 
de incidentes. RID busca proveer un método para comunicar la información 
relevante entre CSIRTs manteniendo la compatibilidad con varios sistemas de 
rastreo y respuesta.

Para conservar la privacidad y tener un buen nivel de seguridad, los mensajes 
RID toman ventaja de los mecanismos provistos por XML. El esquema RID funciona 
como un mecanismo de transporte para soportar la comunicación de documentos 
IODEF para intercambiar y realizar un seguimiento de incidentes de seguridad. 
Los mensajes RID son encapsulados para el transporte, este procedimiento se 
define en el RFC 6046. La autenticación, integridad y autorización son el 
resultado de las capacidades de cada capa y son utilizadas para alcanzar un 
buen nivel de aseguramiento.

Los mensajes RID tienen la intención de ser usados en el manejo coordinado de 
incidentes para localizar la fuente de un ataque y detener o mitigar sus 
efectos. Los tipos de ataque incluyen redes o sistemas que se vieron 
comprometidos, denegaciones de servicio o cualquier otro tipo de tráfico 
malicioso en una red. RID es esencialmente un sistemas de mensaje para coordinar 
la detección de ataques, los mecanismos de rastreo, y el manejo de incidentes. 
Los CSIRTs tienen la posibilidad de reportar ataques que estén ocurriendo a 
otros CSIRTs o pedir información a estos de si también detectaron el ataque. 
Para esto se utiliza el \emph{Report message} para indicar que un reporte a 
sucedido. Con un mensaje de \emph{IncidentQuery} se le puede preguntar a otro 
CSIRT si este ha detectado el ataque.

Los sistemas comprometidos pueden resultar de otros incidentes de seguridad como 
worms, troyanos o virus. %Puede suceder que un incidente no sea reportado incluso 
%cuando la dirección de origen del ataque sea valida y este disponible debido a 
%que sea difícil realizar una acción para mitigar o detener un ataque. 
El manejar 
incidentes es una tarea difícil para un proveedor de internet o en algunos 
clientes debido al tamaño de la red o los recursos disponibles.
RID provee el framework necesario para realizar la comunicación entre redes 
involucradas en el manejo, seguimiento y mitigación de incidentes de seguridad. 
Distintos tipos de mensajes son necesarios para facilitar el manejo de 
incidentes. Los mensajes que se incluyen son \emph{Report}, \emph{Incident Query} 
\emph{TraceRequest}, \emph{RequestAuthorization}, \emph{Result} e 
\emph{Investigation Request message}. El mensaje de \emph{Report} es utilizado 
cuando se ingresa un incidente en un sistema RID y no se deben realizar más 
acciones. Un mensaje \emph{Incident Request} es utilizado para pedir información 
de un incidente en particular. Un mensaje \emph{Trace Request} es utilizado 
cuando la fuente del trafico puede estar oculta. En ese caso, cada proveedor de 
red que reciba uno de estos mensajes enviara uno a la red anterior en el camino 
del mensaje para obtener la fuente del trafico. Los mensajes \emph{Request 
Authorization} y \emph{Result} son utilizados para comunicar el estado y 
resultado de un mensaje \emph{Trace Request} o \emph{Investigation Request}. El 
mensaje \emph{Investigation Request} solo considera los sistemas RID en el 
camino a la fuente del trafico y no el uso de las redes de los sistemas de seguimiento. 
Un mensaje enviado entre sistemas RID para \emph{Trace Request} o 
\emph{Investigation Request} para detener trafico en una fuente por fuera de una 
red debería requerir la siguiente información:
\begin{itemize}
  \item Suficiente información para permitir  a los administradores de la red 
  tomar una decisión sobre la importancia de continuar con el seguimiento.
  \item El incidente o información del paquete IP necesarias para realizar el 
  seguimiento.
  \item Información de contacto o el origen de la comunicación RID. 
  \item Información del camino de la red para prevenir loops en el rutado a 
  través de la red. Si un sistema RID recibe un mensaje de \emph{Trace Request} 
  conteniendo su propia información en el camino el sistema RID debería emitir 
  una alerta.
  \item Un identificador único para cada ataque. Este identificador debería ser 
  utilizado para correlacionar los seguimiento en múltiples fuentes en un 
  ataque DDoS.
\end{itemize}

Es responsabilidad de cada participante adherirse a las reglas establecidas en 
las políticas globales de uso para este sistema y las establecidas en cada 
acuerdo para los acuerdos bilaterales o acordes a las reglas del consorcio. El 
objetivo de dichas políticas es evitar el abuso del sistema.

En RID se pueden diseñar topologías que permitan el intercambio de información 
facilitando la comunicación entre socios. La topología más básica para comunicar sistemas RID es una comunicación directa. 
En una organización se pueden establecer y utilizar varias topologías. Una podría 
fortalecer las relaciones bilaterales con socios. Los socios podrían enrutar los 
mensajes RID entre ellos. Este enfoque permite un rastreo iterativo en donde 
la fuente es desconocida.

La comunicación entre los sistemas RID debe ser protegida. RID tiene muchas 
consideraciones de seguridad incluidas en el diseño del protocolo. Al considerar 
el transporte de mensajes RID, una red fuera de banda, ya sea física o lógica, 
puede prevenir ataques externos contra las comunicaciones RID. Se deben utilizar 
conexiones autenticadas y encriptadas entre sistemas RID para proveer 
confidencialidad, integridad, autenticidad y privacidad de los datos. Las 
relaciones de confianza son realizados por socios y establecen relaciones de 
confianza por medio de infraestructuras de PKI.

El protocolo de transporte utilizado debe proveer encriptación para dar un nivel 
de seguridad e integridad adicional, mientras se provee autenticación por medio 
de certificados. De todas formas los mensajes RID no confían únicamente en la 
seguridad provista por el protocolo de transporte.

La infraestructura PKI provee la base para la autenticación, autorización, 
encriptación y firmas digitales necesarias para establecer una relación de 
confianza entre los miembros de una comunidad RID. 

Problemas de privacidad pueden ser de preocupación cuando se habla de compartir 
información y se deben considerar a la hora de alcanzar la meta de detener o 
mitigar los efectos de un incidente de seguridad. Para ello hay clases 
específicas para automatizar las políticas de privacidad.

Como se explico anteriormente, IODEF describe un formato de documentos XML con 
la finalidad de intercambiar información de seguridad entre CSIRTs u otras 
organizaciones. El formato establecido le permite a los CSIRTs intercambiar 
información que sea fácilmente parseable.

IODEF define un formato de mensaje, no un protocolo de transporte, esto se 
realiza para permitir a los CSIRTs intercambiar y almacenar los datos de la 
forma que más le conviene a cada una de las organizaciones. Sin embargo, RID 
requiere una especificación de un protocolo de transporte para asegurar la 
interoperabilidad entre las organizaciones socias. En el RFC6046, se especifica 
el transporte de mensajes RID sobre HTTP/TLS. En esta especificación, cada 
servidor RID funciona como servidor y cliente. Todos los sistemas RID deben 
estar preparados para aceptar conexiones HTTP/TLS de cualquier socio con la 
finalidad de soportar llamadas en respuestas tardías a pedidos que se realizaron.




