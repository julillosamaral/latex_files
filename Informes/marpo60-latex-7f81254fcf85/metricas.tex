
\section{ Métricas de performance. }

En la siguiente sección se definirán las métricas relevantes para las pruebas de performance del proyecto. Estas métricas serán la base para tomar mediciones y compararlas entre 
distintos sitios. Dada la arquitectura típica de una aplicación web, a continuación se presenta una lista de métricas de interés del lado cliente. Estas métricas son producto de
de la investigación del equipo, algunas de ellas están basadas en las herramientas utilizadas posteriormente para realizar las pruebas.

\begin{itemize}
	\item
	\textbf{Tiempo de carga.} Se define como el tiempo desde el pedido inicial hasta la ocurrencia del evento \emph{window load event} o también conocido como \emph{onload}.
	Básicamente se trata del tiempo entre que el usuario realiza un pedido a un sitio y éste carga completamente. Es importante destacar que esta métrica involucra muchos elementos
	de complejidad alta, como AJAX, carga de contenido de terceros y CDNs.
	\item
	\textbf{Tiempo hasta el primer byte recibido.} Se refiere al tiempo transcurrido desde que el navegador utilizado por el usuario realiza un pedido al servidor, hasta que el
	navegador recibe el primer byte de la respuesta del mismo. Vale la pena aclarar que las respuestas con código HTTP de redirección por parte del servidor web no cuentan como
	respuestas al pedido del cliente. Esta métrica sirve como indicador de la capacidad de respuesta de un servidor web u otros recursos de la red.
	\item
	\textbf{Cantidad de pedidos HTTP.} Cantidad de pedidos HTTP necesarios para descargar todos los elementos de un sitio luego del primer pedido realizado. En este punto es
	importante mencionar la regla de oro de la performance del lado cliente ``realizar la menor cantidad de pedidos HTTP posible''. La idea de obtener mediciones sobre esta métrica es
	comparar el comportamiento del sitio en el caso de un usuario que lo visita por primera vez, y luego vuelve a accederlo. En este segundo acceso vale la pena volver a medir
	la cantidad de pedidos para evaluar la efectividad del \emph{caching} del sitio.
	\item
	\textbf{Uso de conexiones persistentes.} Porcentaje de utilización de la directiva \emph{keep-alive} del protocolo HTTP en caso de que sea necesario descargar varios elementos de
	un mismo \emph{host}. Sin la directiva, se estaría cayendo en una penalización fija de tiempo producida por el tiempo de establecimiento de conexión TCP, el cual consta de un
	costo fijo de tiempo por conexión.
	\item
	\textbf{Porcentaje de uso de compresión}. Porcentaje de respuestas del servidor de tipo texto a las cuales se les aplica algún tipo de compresión por parte del servidor. Mientras
	menos pesada sea la respuesta del servidor, más rápido llegará al navegador cliente.
	\item
	\textbf{Uso eficiente de hojas de estilo y javascripts.} Consta en verificar que se utilice solo una hoja de estilos y un archivo de javascripts. Tener un archivo solo para cada tipo de
	recurso ahorra cantidad de pedidos HTTP.
	\item
	\textbf{Cantidad de redirecciones.} Cantidad de redirecciones hasta el pedido final. Según Steve Souders, las redirecciones son muy ineficientes, ya que implican el paso de un RTT
	para cada uno de los pedidos redireccionados, empeorando la experiencia del usuario final.
	\item
	\textbf{Cantidad de \emph{kilobytes} transferidos}. Cantidad de \emph{kilobytes} transferidos por la red, esta métrica puede medirse por pedido HTTP realizado, o evaluando
	por caso de uso de usuario completo.
\end{itemize}
