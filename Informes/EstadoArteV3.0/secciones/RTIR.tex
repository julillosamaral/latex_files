\subsubsection{Que es RTIR?}
RTIR es un sistema de manejo de incidentes diseñado para ser utilizado por los 
equipos de seguridad de sistemas. A sido creado en conjunto con equipos de CERT 
y CSIRT para manejar el creciente número de incidentes reportados.
Presenta la ventaja de ser opensource, contener una API completa y una comunidad 
de usuarios grande y experta. Además es simple de integrar con otras 
herramientas existentes. Está implementado por medio de módulos PERL y las 
herramientas que provee RT. Se puede pensar en RTIR como una extensión de RT 
para ser utilizada por CERT's y CSIRT's.

Existen algunas alternativas a RTIR como lo son AIRT (Application for Incident Response Teams) 
cuya última versión data de Julio de 2009. AIRT es una aplicación web 
desarrollada para los equipos de respuesta a incidentes. Busca proveer facilidad 
ante los reportes de incidentes de seguridad así como un seguimiento simple de 
estos. Este sistema no cuenta con una comunidad comparable a la de RTIR así como 
con documentación tan extensa como la de RTIR.

Otra de las opciones existentes es OTRS (Open Technology Real Services), así 
como los anteriores también es open soruce. Presenta las ventajas de tener una 
comunidad más numerosa que AIRT y que el código esta siendo desarrollado continuamente. 
Así como RTIR esta desarrollado por medio de Perl y permite conectarse a varias 
bases de datos. Presenta documentación extensa para implementadores.