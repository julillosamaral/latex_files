
\documentclass[11pt]{article}

\usepackage{ifpdf}
\ifpdf 
    \usepackage[pdftex]{graphicx}   % to include graphics
    \pdfcompresslevel=9 
    \usepackage[pdftex,     % sets up hyperref to use pdftex driver
            plainpages=false,   % allows page i and 1 to exist in the same document
            breaklinks=true,    % link texts can be broken at the end of line
            colorlinks=true,
            pdftitle=Actas
            pdfauthor=Julio Saráchaga           ]{hyperref} 
    \usepackage{thumbpdf}
\else 
    \usepackage{graphicx}       % to include graphics
    \usepackage{hyperref}       % to simplify the use of \href
\fi 
\RequirePackage[spanish]{babel}
\RequirePackage[utf8]{inputenc}
\title{Acta reunión 11 de Noviembre 2013}
\author{Julio Saráchaga}
\date{}

\begin{document}
\maketitle

\begin{itemize}
  \item Pensar en la completitud de los casos de uso en la especificación.
  \item Porque la organización decidió utilizar RTIR.
	  \subitem - La organización ya lo usa, esta instalada y funcionando adecuadamente
	  \subitem - Decir que se pueda integrar pero no casarse con el RTIR.
	  \subitem - Mejorar
 \item Porque la utilización del modelo.
	\subitem - No es una organización de gobierno
	\subitem - Tampoco es una organización coordinadora de esfuerzos.
	\subitem - Enfocar contacto directo con otros CSIRTs y es una organización de investigación.
	\subitem - Mejorar
  \item Armar una matriz de excel, poner requerimientos y casos de uso y cuales contribuyen a satisfacer los requerimientos.
  \item Respecto a los requerimientos:
    \subitem - En el requerimiento funcional no tienen porque ser solo los intercambiados.
  \item Poner el alta de las políticas y modificación como caso de uso.
  \item Poner las políticas de sanitización en el diagrama de secuencia.
  \item Empezar a trabajar con el diseño.
\end{itemize}

\end{document}  
